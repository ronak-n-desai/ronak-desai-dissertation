\chapter{Computational Methods} \label{ch:3}

\section{The Particle-In-Cell Method}

The Particle-In-Cell (PIC) method involves solving Maxwell's Equations on a grid 

\begin{align}
	\nabla \cdot \vec{E} &= \frac{\rho}{\epsilon_0}  \label{eq:gauss} \\
	\nabla \times \vec{E} &= - \frac{\partial \vec{B}}{\partial t} \label{eq:faraday} \\
	\nabla \cdot \vec{B} &= 0 \\
	\nabla \times \vec{B} &= \mu_0 (\vec{J} + \epsilon_0 \frac{\partial \vec{E}}{\partial t}) \label{eq:ampere}
\end{align}
In practice, the relativistic versions of these equations are used, but the concepts remain the same. 

\subsection{Field Solver: FDTD}

\subsection{Particle Push: Boris Algorithm}

\begin{equation}
	\frac{v_{n + 1/2} - v_{n - 1/2}}{\Delta t} = \frac{q}{m}[E_n + v \times B]
\end{equation}

\subsection{EPOCH Code}

\begin{quote}
	Clearly Top-hat shape functions should never be used for laser-solid simulations.
\end{quote}

\subsection{Convergence Testing the EPOCH Code}


\section{Machine Learning}

\subsection{Simple Models}

\subsubsection{Polynomial Regression}

\subsubsection{Support Vector Regression}


\subsection{Advanced Models}

\subsubsection{Gaussian Process}

\subsection{Neural Networks}