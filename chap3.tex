\chapter{Computational Methods} \label{ch:3}

\section{The Particle-In-Cell Method}

The Particle-In-Cell (PIC) method involves solving Maxwell's Equations on a grid 

\begin{align}
	\nabla \cdot E &= \frac{\rho}{\epsilon_0}  \label{eq:gauss} \\
	\nabla \times \vec{E} &= - \frac{\partial \vec{B}}{\partial t} \label{eq:faraday} \\
	\nabla \cdot \vec{B} &= 0 \\
	\nabla \times \vec{B} &= \mu_0 (\vec{J} + \epsilon_0 \frac{\partial \vec{E}}{\partial t}) \label{eq:ampere}
\end{align}
In practice, the relativistic versions of these equations are used, but the concepts remain the same. 

\section{Machine Learning}

\subsection{Polynomial Regression}

\subsection{Gaussian Process}

\subsection{Neural Network}