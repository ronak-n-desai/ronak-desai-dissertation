\chapter{My First Chapter}
\begin{quote}
\lipsum[1]
\end{quote}

\lipsum[1-3]


\section{Glossaries}
Glossaries are disabled in this version of the template - if you are interested get osudiss-template\_glossary.
The latex glossaries package can be useful for keeping track of your acronyms, and making a nice hyperlinked 
list at the beginning of your document.  Note: a glossary/list of acronyms is NOT required by the GS.
If you do wish to include one, it \textbf{must} appear directly after the lists of figures and tables.
To use the glossaries package, you must load it in your preamble:
\begin{verbatim}
\usepackage[acronym, section=chapter]{glossaries} %load package
\makeglossaries %required to actually make a glossary
%A list of common acronyms
%Only those used will be displayed, so you can just add to this list
\newacronym{LLNL}{LLNL}{Lawrence Livermore National Laboratory}
\newacronym{LANL}{LANL}{Los Alamos National Laboratory}
\newacronym{NIF}{NIF}{National Ignition Facility}
\newacronym{NIF-ARC}{NIF-ARC}{National Ignition Facility - Advanced Radiographic Capability}
\newacronym{OMEGA-EP}{OMEGA-EP}{OMEGA Extended Performance System}
\newacronym{RF}{RF}{radio frequency}
\newacronym{QCD}{QCD}{quantum chromodynamics}
\newacronym{QED}{QED}{quantum electrodynamics}
\newacronym{WKB}{WKB}{Wentzel-Kramers-Brillouin}
\newacronym{SRIM}{SRIM}{stopping range of ions in matter}
\newacronym{SOBP}{SOBP}{spread-out Bragg peak}
\newacronym{OSUCCC}{OSUCCC}{The Ohio State Comprehensive Cancer Center}
\newacronym{LLE}{LLE}{Laboratory for Laser Energetics}
\newacronym{TNSA}{TNSA}{target normal sheath acceleration}
\newacronym{IBA}{IBA}{Ion Beam Analysis}
\newacronym{IMPT}{IMPT}{intensity modulated proton therapy}
\newacronym{ALLS}{ALLS}{Advanced Laser Light Source}
\newacronym{XPIF}{XPIF}{x-ray and particle-induced fluorescence}
\newacronym{EDX}{EDX}{energy dispersive x-ray fluorescence}
\newacronym{TEM}{TEM}{transverse electromagnetic mode}
\newacronym{FWHM}{FWHM}{full width at half maximum}
\newacronym{eTNSA}{eTNSA}{enhanced target normal sheath acceleration}
\newacronym{LWFA}{LWFA}{laser wakefield acceleration}
\newacronym{PIC}{PIC}{particle-in-cell}
\newacronym{NGP}{NGP}{nearest grid point}
\newacronym{CCD}{CCD}{charge coupled device}
\newacronym{RPA}{RPA}{radiation pressure acceleration}
\newacronym{LSP}{LSP}{Large Scale Plasma: An implicit particle-in-cell code}
\newacronym{NN}{NN}{neural network model}
\newacronym{SVGP}{SVGP}{stochastic variational gaussian process}
\newacronym{GPR}{GPR}{gaussian process regression}
\newacronym{POLY}{POLY}{polynomial regression}
\newacronym{ML}{ML}{machine learning}
\newacronym{AI}{AI}{artificial intelligence}
\newacronym{HEDS}{HEDS}{high energy density science}
\newacronym{BO}{BO}{bayesian optimization}
\newacronym{WP-ELL}{WP-ELL}{Extreme Light Laboratory at the Wright-Patterson Air Force Base}
\newacronym{GPU}{GPU}{Graphics Processing Unit}
\newacronym{OSC}{OSC}{Ohio Supercomputer Center}
\newacronym{SVR}{SVR}{Support Vector Regression}
\newacronym{RBF}{RBF}{Radial Basis Function}
\newacronym{MAPE}{MAPE}{Mean Absolute Percentage Error}
\newacronym{MSE}{MSE}{Mean Squared Error}
\newacronym{RMSE}{RMSE}{Root Mean Squared Error}
\newacronym{AFIT}{AFIT}{Air Force Institute of Technology}
\newacronym{CSUCI}{CSUCI}{California State University - Channel Islands}
\newacronym{CSU}{CSU}{Colarado State University}
\newacronym{RAL}{RAL}{Rutherford Appleton Laboratory}
\newacronym{DAQ}{DAQ}{Data Acquisition System}
\newacronym{OAP}{OAP}{off-axis parabolic mirror}
\newacronym{HDF5}{HDF5}{Hierarchical Data Format 5}
\newacronym{EPICS}{EPICS}{Experimental Physics and Industrial Control System}
\newacronym{GUI}{GUI}{Graphical User Interface}
 %load list of acronyms contained in acronyms.tex
\end{verbatim}
Then, to define an acronym, you will need to include it in your preamble:
\begin{verbatim}
\newacronym{AFM}{AFM}{atomic force microscopy}
\end{verbatim}
In this template, the acronyms are all placed in a separate file called
\verb#acronmys.tex# which is then loaded with \verb#%A list of common acronyms
%Only those used will be displayed, so you can just add to this list
\newacronym{LLNL}{LLNL}{Lawrence Livermore National Laboratory}
\newacronym{LANL}{LANL}{Los Alamos National Laboratory}
\newacronym{NIF}{NIF}{National Ignition Facility}
\newacronym{NIF-ARC}{NIF-ARC}{National Ignition Facility - Advanced Radiographic Capability}
\newacronym{OMEGA-EP}{OMEGA-EP}{OMEGA Extended Performance System}
\newacronym{RF}{RF}{radio frequency}
\newacronym{QCD}{QCD}{quantum chromodynamics}
\newacronym{QED}{QED}{quantum electrodynamics}
\newacronym{WKB}{WKB}{Wentzel-Kramers-Brillouin}
\newacronym{SRIM}{SRIM}{stopping range of ions in matter}
\newacronym{SOBP}{SOBP}{spread-out Bragg peak}
\newacronym{OSUCCC}{OSUCCC}{The Ohio State Comprehensive Cancer Center}
\newacronym{LLE}{LLE}{Laboratory for Laser Energetics}
\newacronym{TNSA}{TNSA}{target normal sheath acceleration}
\newacronym{IBA}{IBA}{Ion Beam Analysis}
\newacronym{IMPT}{IMPT}{intensity modulated proton therapy}
\newacronym{ALLS}{ALLS}{Advanced Laser Light Source}
\newacronym{XPIF}{XPIF}{x-ray and particle-induced fluorescence}
\newacronym{EDX}{EDX}{energy dispersive x-ray fluorescence}
\newacronym{TEM}{TEM}{transverse electromagnetic mode}
\newacronym{FWHM}{FWHM}{full width at half maximum}
\newacronym{eTNSA}{eTNSA}{enhanced target normal sheath acceleration}
\newacronym{LWFA}{LWFA}{laser wakefield acceleration}
\newacronym{PIC}{PIC}{particle-in-cell}
\newacronym{NGP}{NGP}{nearest grid point}
\newacronym{CCD}{CCD}{charge coupled device}
\newacronym{RPA}{RPA}{radiation pressure acceleration}
\newacronym{LSP}{LSP}{Large Scale Plasma: An implicit particle-in-cell code}
\newacronym{NN}{NN}{neural network model}
\newacronym{SVGP}{SVGP}{stochastic variational gaussian process}
\newacronym{GPR}{GPR}{gaussian process regression}
\newacronym{POLY}{POLY}{polynomial regression}
\newacronym{ML}{ML}{machine learning}
\newacronym{AI}{AI}{artificial intelligence}
\newacronym{HEDS}{HEDS}{high energy density science}
\newacronym{BO}{BO}{bayesian optimization}
\newacronym{WP-ELL}{WP-ELL}{Extreme Light Laboratory at the Wright-Patterson Air Force Base}
\newacronym{GPU}{GPU}{Graphics Processing Unit}
\newacronym{OSC}{OSC}{Ohio Supercomputer Center}
\newacronym{SVR}{SVR}{Support Vector Regression}
\newacronym{RBF}{RBF}{Radial Basis Function}
\newacronym{MAPE}{MAPE}{Mean Absolute Percentage Error}
\newacronym{MSE}{MSE}{Mean Squared Error}
\newacronym{RMSE}{RMSE}{Root Mean Squared Error}
\newacronym{AFIT}{AFIT}{Air Force Institute of Technology}
\newacronym{CSUCI}{CSUCI}{California State University - Channel Islands}
\newacronym{CSU}{CSU}{Colarado State University}
\newacronym{RAL}{RAL}{Rutherford Appleton Laboratory}
\newacronym{DAQ}{DAQ}{Data Acquisition System}
\newacronym{OAP}{OAP}{off-axis parabolic mirror}
\newacronym{HDF5}{HDF5}{Hierarchical Data Format 5}
\newacronym{EPICS}{EPICS}{Experimental Physics and Industrial Control System}
\newacronym{GUI}{GUI}{Graphical User Interface}
#, to
make the main source file easier to read.  You can also just put all
your acronyms directly in the header of your main source file.

Once you have defined an acronym, you can use it with the \verb#\gls{<label>}# command.
The first time you use the acronym, the full definition is printed: atomic force microscopy (AFM). %\gls{AFM}.
On subsequent uses, just the abbreviation is printed: AFM %\gls{AFM}
 - Latex keeps track
for you, so you don't have to do this manually.  There are fancier forms
of \verb#\gls# that allow you to capitalize (\verb#\Gls{<label>}#) or use the 
plural (\verb#\glspl{<label>}#) forms of your abbreviations.  See 
for example \url{http://en.wikibooks.org/wiki/LaTeX/Glossary} for details
on the glossaries package.

If you want to list all your acronyms at the beginning of your document, you
will need to include the \verb#\makeglossaries# command in your preable, as
shown above, and then 
\begin{verbatim}
\printglossary[type=\acronymtype]
\end{verbatim}
where you want the list of abbreviations to actually appear 
(as stated above, this must be right after your lists of figures and tables in the roman
numeral pages at the beginning of the document).  Then you have
to go to the terminal (in the directory containing your document), and run the command
\begin{verbatim}
makeglossaries (yourdocumentname)
\end{verbatim}
to actually create the glossary.  If you just want Latex to keep track of your acronyms for you 
and print no list, you can just skip this step and the \verb#printglossary# command.

To make the heading for the List of Abbreviations look the same as the List of Figures
and List of Tables, instead of using \verb#\printglossary# I defined a custom command to print the glossary:
\begin{verbatim}
\newcommand\PrintListofAbbreviations[1]{
\phantomsection
\addcontentsline{toc}{front}{\typesetColumnHeading{#1}}
\printglossary[type=\acronymtype,title={\protect {\typesetLevelTwo{#1}}}]
\end{verbatim}
This command functions the same as \verb#\printglossary#, but instead
of typesetting the heading like a chapter heading (in smallcaps), 
it typesets in bold like the lists of figures and tables.  
To use it, just call \verb#\PrintListofAbbreviations{List of Abbreviations}#
right after \verb#\listoftables# and before \verb#\mainmatter#.
If you wish to use a different title for you list of abbreviations,
just call \verb#\PrintListofAbbreviations{Your Title}#.
I am unsure if the GS has any restrictions on what title you can use.

One final comment, this template uses \verb#makeindex# to create the glossaries
for compatability (some versions of Ubuntu for example don't support \verb#xindy#).
However, there is a better program for making glossaries that you can use by calling
\begin{verbatim}
\usepackage[xindy,toc,acronym, section=chapter]{glossaries}
\end{verbatim}
instead.  In particular, \verb#xindy# allows you to use symbols in your abbreviations,
such as Greek letters or accented characters, which are not supported by \verb#makeindex#.


\section{Example Figures} %this figure requires \usepackage{subfig}
 \begin{figure}
 \centering

 %The actual figure
 \includegraphics[width=0.45\textwidth]{sine}

 \caption[My first figure]%the text in [] will be displayed in the list of figures.  
  %If this is omitted, the full figure caption (follows in {}) will be displayed in the list of figures.
  {\label{fig:sine} My first figure, showing a plot of the sine function.}
 \vspace{0.5 in}
 \end{figure}

Fig.\ref{fig:sine} is produced by the following
\begin{verbatim}
 \begin{figure}
 \includegraphics[width=0.45\textwidth]{sine}
 \caption[My first figure] 
  {\label{fig:sine} My first figure, showing a plot of the sine function.}
 \end{figure}
\end{verbatim}
The text \verb#[My first figure]# will be displayed in the list of figures.
If this is omitted, then the full figure caption (follows in \verb#{}#) 
will be displayed.  If you compile using \verb#latex#, then \verb#sine.eps# will be included.
If you compile using \verb#pdflatex#, then \verb#sine.pdf# will be used.  \verb#pdflatex#
also supports including \verb#.jpg# files if you prefer, but there can be artifacts when 
\verb#.jpg# files are rescaled, however one cannot always generate a PDF if an
image is ``borrowed'' (Fig.~\ref{fig:battery} uses jpegs).  
Note that you do \textbf{not} need to include a file extension 
for the figure - latex adds the appropriate extension automatically.  If you do
not want to place your figure in the same directory as the latex root document 
(in this case \verb#template.txt#), then you could instead use something like:
\begin{verbatim}
 \includegraphics[width=0.45\textwidth]{directory/figure}
\end{verbatim}

 \begin{figure}[h]
 \centering
\subfloat[Eating One Battery]{\label{one_battery}
 %The actual figure, part (a)
 \includegraphics[width=0.45\textwidth]{one_battery}}
\hspace{0.5 in}
\subfloat[Eating Five Batteries]{\label{five_battery}
 %The actual figure, part (a)
 \includegraphics[width=0.3\textwidth]{five_battery}}


 \caption[Eating Batteries]%the text in [] will be displayed in the list of figures.  
  %If this is omitted, the full figure caption (follows in {}) will be displayed in the list of figures.
  {\label{fig:battery} A man eating batteries \cite{English}.  
  Reprinted with permission from Strong Bad.}
 \vspace{0.5 in}
 \end{figure}

Fig.\ref{fig:battery} is produced by the following
\begin{verbatim}
 \begin{figure}
\subfloat[Eating One Battery]{
\includegraphics[width=0.45\textwidth]{one_battery}}
\subfloat[Eating Five Batteries]{
 \includegraphics[width=0.3\textwidth]{five_battery}}
 \caption[Eating Batteries]
  {\label{fig:battery} A man eating batteries. Reprinted with permission from Strong Bad.}
 \end{figure}
\end{verbatim}
This figure is a reproduction of a figure from another ``work''.  
In this case the ``author'' gives explicit permission to reuse this figure,
but in most cases you will need to contact the journal to obtain permission to 
reproduce any part of an article, such as a figure.  
Journals are generally very nice about letting you reproduce figures 
for a thesis, and give permission very rapidly (sometime instantly online)
and don't charge anything.  However, if you do not obtain permission
you can have serious trouble after your thesis is published.
Note that you need to obtain permission to reuse figures
from any published paper, 
\textit{even if you are the author.} 

This figure also illustrates the use of subfigures via the subfig package, loaded by placing 
\verb#\usepackage{subfig}# in the preamble of \verb#template.txt#.  This allows you to include
each subfigure as an individual image, and have latex arrange and label them for you 
(including an optional name for each subfigure in \verb#[]#).  This also allows
you to individually reference Fig.~\ref{one_battery} and Fig.~\ref{five_battery}.
Note that many journals DO NOT support this, and require that each figure be a single
image, but this should be fine for a thesis.  


\section{This is a Section!}

\lipsum[2-4]
\subsection{This is a Subsection?}

\lipsum[5-10]

\subsubsection{This is a Subsubsection.}

\lipsum

\section{Another Section}

\lipsum[2-4]

\begin{table}
\begin{center}
\begin{tabular}{lll}
\hline
\\[3pt]
Alphabet Character & Vowel & Number \\[3pt]
\hline
\\[3pt]
A & Yes & 1 \\
B & No & 2 \\
C & No & 3 \\[3pt]
\hline
\end{tabular}
\label{tbl1}
\caption{My first Table}
\end{center}
\end{table}

\lipsum[4-6]
\begin{table}
\begin{center}
\begin{tabular}{lll}
\hline
\\[3pt]
Alphabet Character & Vowel & Number \\[3pt]
\hline
\\[3pt]
A & Yes & 1 \\
B & No & 2 \\
C & No & 3 \\[3pt]
\hline
\end{tabular}
\label{tbl2}
\caption{My second Table}
\end{center}
\end{table}

\lipsum[1-12]
\begin{equation}
\partial_\mu F^{\mu\nu} = J^\nu
\end{equation}
Lorem ipsum your face. Lorem ipsum your face. 

\lipsum[1]A footnote\footnote{Another foot.}.\lipsum[2-5]
\begin{equation}
E = \gamma mc^2
\end{equation}
Lorem ipsum your face. Lorem ipsum your face. 

\lipsum[2-5]
