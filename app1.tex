\chapter{Energy Conservation in EPOCH Particle-in-Cell Simulations Due to Finite Numbers of Particles}
This appendix focuses on unpublished work that was done jointly with Ricky Oropeza and Joseph Smith. My contributions to this project were primarily done as a pre-candidacy student.

Particle-in-Cell (PIC) simulations provide a useful but imperfect model of various plasma phenomena. In this work, the impact of the finite number of particles in a PIC simulation on the energy conservation is considered and explored through ultra-intense laser interactions with a thin, near solid density target in the Target Normal Sheath Acceleration (TNSA) regime. 

\section{Background}

Explicit PIC codes tend to gain energy over time through what can be attributed as numerical errors. In this section, we consider which plasma and simulation parameters affect this numerical energy gain and derive various scalings. 

\subsection{Electric Field Fluctuations}

In PIC simulations, we compute the velocities at the next timestep through \cref{eq:v_update} which is dependent on the time-step $\Delta t$. Due to using a finite grid, approximating real particles with macro particles, and using a finite $\Delta t$, we will develop some errors in calculating the electric field $\delta E$. The corresponding force miscalculation $\delta F = q \delta E \Delta t$ would deliver an impulse $m \delta v$ and result in a velocity difference \cite{Hockney_1988_PIC} of

\begin{equation}
	\delta v = \frac{q}{m} \Delta t \delta E
\end{equation}
We can make an assumption that these field calculation errors will be randomly distributed which can be treated as a random walk in velocity space. If we consider $\Delta v$ as the total deviation of the calculated velocity from the true value, we should expect $\langle \Delta v \rangle = 0$ due to the symmetry of this random walk. However, the squared deviations on average will increase over time; for $n$ time-steps (each with the same random error $\delta E$), we would have 

\begin{equation}
	\langle \Delta v^2 \rangle = n \delta v ^2 = n \frac{q^2}{m^2} \Delta t^2 \delta E^2
\end{equation}
We can see that the average change in kinetic energy $\Delta KE \equiv \frac{1}{2} m \langle v^2 \rangle$ increases linearly with the number of time-steps $n$ \cite{Hockney_1988_PIC}. Additionally, since $\Delta KE \propto \frac{1}{m}$, the heavier particles (i.e. ions) can usually be neglected when examining the artificial heating \cite{Hockney_1988_PIC}. Hockney postulates a related expression \cite{Hockney_1971_JoCP} for $\Delta KE$ in another work as 

\begin{equation}
	\Delta KE \sim \frac{q^2}{m} \langle E^2 \rangle \tau_\text{corr} \Delta t
\end{equation}
where $\tau_\text{corr}$ can be identified with the period of plasma oscillations $\sim \omega_{p,e}^{-1}$. Then, expressing the charge of one electron macro-particle as $q = e \frac{N}{N_{mac}} = \frac{e n}{n_{mac}} = \frac{e n \Delta x^2}{n_{ppc}}$, where $\Delta x$ is the cell size and $n_\text{ppc}$ is the number of electron macro-particles per cell, the kinetic energy increase becomes

\begin{equation}
	\Delta KE = (\frac{e}{m_e}) \frac{e n \Delta x^2}{n_{ppc}} \langle E^2 \rangle \frac{2 \pi}{\omega_{pe}} \Delta t 
\end{equation}
Hockney uses a result from Chapter 8.2 of Montgomery and Tidman \cite{Montgomery_1964_Plasma} for the squared electric field fluctuations

\begin{equation} 
	\frac{\langle E^2 \rangle}{8 \pi} = \frac{k_B T}{2} \int \int_{-\infty}^{+\infty} \frac{d k_x d k_y}{(2 \pi)^2} \frac{1}{(1 + (k_x^2 + k_y^2) \lambda_D^2)} 
\end{equation}
which can be solved by letting $u = k \lambda_D$ where $k = \sqrt{k_x^2 + k_y^2}$ and integrating with respect to the polar area element $k \; dk \; d \phi$

\begin{equation}
	\langle E^2 \rangle = \frac{k_B T}{4 \pi \epsilon_0 \lambda_D^2} \text{Log}(1 + u_{max^2}) \label{eq:e2}
\end{equation}
Here, $u_{max} = k_\text{max} \lambda_D$ corresponds to the maximum wavenumber $k_{max} = \frac{2 \pi}{\Delta x}$ considered which is limited by the resolution. 

\subsection{Empirical Heating Estimates}

Using \cref{eq:debye,eq:omegape}, $\Delta KE$ can now be expressed as 

\begin{equation}
	\Delta KE = \frac{n_e^{3/2} \Delta x^2}{n_{ppc}} \text{Log}(1 + u_{max}^2) \label{eq:logarber}
\end{equation}
A more empirical estimate for the heating can be obtained by asserting a general scaling of the heating time $\tau_H \simeq \frac{n_\text{ppc}^\alpha}{\omega_{p,e}} \left(\frac{\lambda_{D,0}}{\Delta x}\right)^d$, where $d$ and $\alpha$ are constants that can be fit empirically through simulations. If we assert that the linear energy increase $\frac{dT}{dt} = T_0 / \tau_H$, we develop a formula (again using \cref{eq:debye,eq:omegape}) for the linear energy increase 

\begin{equation}
	\frac{dT_{eV}}{dt_{ps}} = C_H \frac{T_{0,eV}^{1 - d/2} \Delta x_{nm}^d n_{23}^{(d + 1)/2}}{n_\text{ppc}^\alpha} \label{eq:generalizedarber}
\end{equation}
and when $\alpha = 1$ and $d = 2$, we obtain eq. (30) from Arber et al. \cite{Arber_2015_PPCF}

\begin{equation}
	\frac{dT_{eV}}{dt_{ps}} = C_H \frac{\Delta x_{nm}^2 n_{23}^{3/2}}{n_\text{ppc}} \label{eq:arber}
\end{equation}
where $C_H$ is a constant determined by the shape function and the use of current smoothing. The cell size, number density, time, and temperature are expressed in nm, $10^{23} \text{cm}^{-3}$, ps, and eV due to being convenient units for PIC simulations. A more sophisticated empirical model could also account for two dimensionless timescales: $\omega_{p, e} \Delta t$ and $v_\text{th} \Delta t$, but Hockney \cite{Hockney_1971_JoCP} notes that these can be ignored by constraining $\omega_{p,e} \Delta t$ to be $\omega_{p,e} \Delta t = \text{min}((2 \lambda_D / \Delta x)^{-1}, 1)$ \cite{Hockney_1971_JoCP}.

\section{Methods}

\section{Conclusion}
