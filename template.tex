%%%%%%%%%%%%%%%%%%%%%%%%%%%%%%%%%%%%%%%%%%%%%%%%%%%%%%%%%%%%%%%%%%%%%%%%%%%%%%%
% template.tex - version 0.9.1.1 (5/26/2011)
%
% This is a template file for the osudiss-2 class. See
% osudiss-2.pdf for documentation, and the GS material 
% for the requirements.
%
% Copy the following osudiss-2 files your latex path (or just the folder containing this file):
% osudiss-2.cls (v0.9.1)
% sa-draftwater.sty
%
% Then, to compile this file:
% latex template
% bibtex template
% latex template
% latex template
%
% (You can also use pdflatex if you prefer.)
%
%%%%%%%%%%%%%%%%%%%%%%%%%%%%%%%%%%%%%%%%%%%%%%%%%%%%%%%%%%%%%%%%%%%%%%%%%%%%%%%
\documentclass[11pt, draft, onehalf, phd]{osudiss-2} 
% The `11pt' option is unnecessary since it is the default

% `onehalf' sets the line spacing to one-and-a-half spacing instead of
% double spacing.

% The `phd' option is unnecessary since it is the default

% Remove `draft' option for final draft

%%%%%%%%%%%%%%%%%%%%%%%%% Packages %%%%%%%%%%%%%%%%%%%%%%%%%
% Load your favorite packages here
\usepackage{graphicx} % for importing images in figures - you definitely want this!
\usepackage{lipsum} % for fake latin text---you probably don't want this

% For instance... see osudiss-2.pdf for some suggestions, if you don't
% have a clue
\usepackage{bm} % for bold math---useful
\usepackage{booktabs} % for more professional tables

%hyperref packages and options
\usepackage{bookmark} % helps booksmarks look better in PDF
%hypersetup option 'breaklinks' is reguired for line wrapping in the table of contents during latex compilation, and can be removed if you use pdflatex
\hypersetup{colorlinks=true,linkcolor=blue, breaklinks} %internal links in blue, citations in green
%\hypersetup{colorlinks=true,linkcolor=black, citecolor=black, breaklinks} %all links in black
\usepackage[all]{hypcap}

%Use of natbib is STRONGLY recommended to sort and compress your references within each citation
%With these options, natbib will convert i.e. [5,3,9,4] to [3-5, 9]
\usepackage[sort&compress]{natbib}

%required to have latex automatically generate subfigures (i.e. (a), (b) etc)
\usepackage{subfig}

%load glossaries packages
\usepackage[acronym, section=chapter]{glossaries}
%\usepackage[xindy,acronym, section=chapter]{glossaries} - recommended if supported by your OS
\makeglossaries %required to actually make a glossary
%A list of common acronyms
%Only those used will be displayed, so you can just add to this list
\newacronym{LLNL}{LLNL}{Lawrence Livermore National Laboratory}
\newacronym{LANL}{LANL}{Los Alamos National Laboratory}
\newacronym{NIF}{NIF}{National Ignition Facility}
\newacronym{NIF-ARC}{NIF-ARC}{National Ignition Facility - Advanced Radiographic Capability}
\newacronym{OMEGA-EP}{OMEGA-EP}{OMEGA Extended Performance System}
\newacronym{RF}{RF}{radio frequency}
\newacronym{QCD}{QCD}{quantum chromodynamics}
\newacronym{QED}{QED}{quantum electrodynamics}
\newacronym{WKB}{WKB}{Wentzel-Kramers-Brillouin}
\newacronym{SRIM}{SRIM}{stopping range of ions in matter}
\newacronym{SOBP}{SOBP}{spread-out Bragg peak}
\newacronym{OSUCCC}{OSUCCC}{The Ohio State Comprehensive Cancer Center}
\newacronym{LLE}{LLE}{Laboratory for Laser Energetics}
\newacronym{TNSA}{TNSA}{target normal sheath acceleration}
\newacronym{IBA}{IBA}{Ion Beam Analysis}
\newacronym{IMPT}{IMPT}{intensity modulated proton therapy}
\newacronym{ALLS}{ALLS}{Advanced Laser Light Source}
\newacronym{XPIF}{XPIF}{x-ray and particle-induced fluorescence}
\newacronym{EDX}{EDX}{energy dispersive x-ray fluorescence}
\newacronym{TEM}{TEM}{transverse electromagnetic mode}
\newacronym{FWHM}{FWHM}{full width at half maximum}
\newacronym{eTNSA}{eTNSA}{enhanced target normal sheath acceleration}
\newacronym{LWFA}{LWFA}{laser wakefield acceleration}
\newacronym{PIC}{PIC}{particle-in-cell}
\newacronym{NGP}{NGP}{nearest grid point}
\newacronym{CCD}{CCD}{charge coupled device}
\newacronym{RPA}{RPA}{radiation pressure acceleration}
\newacronym{LSP}{LSP}{Large Scale Plasma: An implicit particle-in-cell code}
\newacronym{NN}{NN}{neural network model}
\newacronym{SVGP}{SVGP}{stochastic variational gaussian process}
\newacronym{GPR}{GPR}{gaussian process regression}
\newacronym{POLY}{POLY}{polynomial regression}
\newacronym{ML}{ML}{machine learning}
\newacronym{AI}{AI}{artificial intelligence}
\newacronym{HEDS}{HEDS}{high energy density science}
\newacronym{BO}{BO}{bayesian optimization}
\newacronym{WP-ELL}{WP-ELL}{Extreme Light Laboratory at the Wright-Patterson Air Force Base}
\newacronym{GPU}{GPU}{Graphics Processing Unit}
\newacronym{OSC}{OSC}{Ohio Supercomputer Center}
\newacronym{SVR}{SVR}{Support Vector Regression}
\newacronym{RBF}{RBF}{Radial Basis Function}
\newacronym{MAPE}{MAPE}{Mean Absolute Percentage Error}
\newacronym{MSE}{MSE}{Mean Squared Error}
\newacronym{RMSE}{RMSE}{Root Mean Squared Error}
\newacronym{AFIT}{AFIT}{Air Force Institute of Technology}
\newacronym{CSUCI}{CSUCI}{California State University - Channel Islands}
\newacronym{CSU}{CSU}{Colarado State University}
\newacronym{RAL}{RAL}{Rutherford Appleton Laboratory}
\newacronym{DAQ}{DAQ}{Data Acquisition System}
\newacronym{OAP}{OAP}{off-axis parabolic mirror}
\newacronym{HDF5}{HDF5}{Hierarchical Data Format 5}
\newacronym{EPICS}{EPICS}{Experimental Physics and Industrial Control System}
\newacronym{GUI}{GUI}{Graphical User Interface}
 %load list of acronyms contained in acronyms.tex

%The following commands can be used to help deal with "overfull hbox" issues
%See, for example, http://www.tex.ac.uk/cgi-bin/texfaq2html?label=overfull for details
%\pretolerance 1000
\setlength{\emergencystretch}{3em}
%\tolerance 1000

%%%%%%%%%%%%%%%%%%%%%%%%% Custom Commands/Environments %%%%%%%%%%%%%%%%%%%%%%%%%
% Put your favorite custom commands here
\newcommand{\fish}{\alpha} % some of my students call it the "fish" symbol

%Print list of abbreviations - use same font as List of Figures and List of Tables for the title, and same formatting in the table of contents.
% Argument #1 - title for list of abbreviations (i.e. List of Abbreviations)

\newcommand\PrintListofAbbreviations[1]{
\phantomsection
\addcontentsline{toc}{front}{\typesetColumnHeading{#1}}
\printglossary[type=\acronymtype,title={\protect {\typesetLevelTwo{#1}}}]
}

% Below is an example of customizing the style of headings in your
% dissertation. See osudiss-2.pdf for more information.
%
% For example, if you simply must have uppercase titles:
%\renewcommand\typesetLevelOne[1]{{\Large\textbf{\MakeUppercase{#1}}\par}}
%\renewcommand\typesetLevelTwo[1]{{\Large\textbf{\MakeUppercase{#1}}}}
% Note the \par for \typesetLevelOne
%
% If you want the title to be bold and |\Large| instead of |\Huge|:
%\renewcommand\titleFont{\normalfont\Large\bfseries}

% Add words that TeX may not know how to hyphenate below. This can
% help prevent overfull hboxes. For example,
\hyphenation{eigen-state space-time} 

%%%%%%%%%%%%%%%%%%%%%%%%% Document Metadata %%%%%%%%%%%%%%%%%%%%%%%%%
\title{Dissertation Title}
\author{Your name}
\advisorname{Professor Your advisor}
\degree{Doctor of Philosophy} % Default value
\member{Some Joe \#1}
\member{Some Jane \#2}
\member{Some other Person \#3}
\authordegrees{B.S.}
\graduationyear{2969}
\unit{Graduate Program in Procrastination} % defaults to ``Graduate Program in Physics''

%%%%%%%%%%%%%%%%%%%%%%%%% Begin Document %%%%%%%%%%%%%%%%%%%%%%%%%
\begin{document}

\frontmatter

\begin{abstract}
An abstract goes here. It should be less than \textbf{500 words}.
\end{abstract}

\dedication{To you, my love.} % Optional, and seriously not this lame
\begin{acknowledgments}
I would like to thank my meemaw and peepaw.
\end{acknowledgments}

\begin{vita}
\dateitem{February 31, 1969}{Born---Crazytown, OH}
\dateitem{Mayuary, 2000}{B.S., Party College, Party Town, Party State}
% Insert other relevant items here (GTA, etc.)

\begin{publist}
\pubitem{Paper 1}
\pubitem{Paper 2}
\end{publist}

\begin{fieldsstudy}
\majorfield{Physics}
\onestudy{small pendulums}{my advisor's name} % optional
% Alternatively you can do:
% \begin{studieslist}
% \studyitem{Topic 1}{Professor 1}
% \studyitem{Topic 2}{Professor 2}
% \studyitem{Topic 3}{Professor 3}
% \end{studieslist}
\end{fieldsstudy}

\end{vita}

\tableofcontents 

% list of figures (comment out if you don't have any figures)
\clearpage %remove if you don't want a page break before list of figures
\listoffigures 

% list of tables (comment out if you don't have any tables)
\clearpage  %remove if you don't want a page break before list of tables
\listoftables 

%print glossary - comment out if you don't want this.  Make sure you also add \glsdisablehyper if you don't want to print a glossary, but do use the %glossaries package to keep track of acronyms
\clearpage %remove if you don't want a page break before list of abbreviations
\PrintListofAbbreviations{List of Abbreviations} %Title is in { } - change if desired

\mainmatter
\chapter{Introduction}

%\begin{quote}
%\lipsum[1]
%\end{quote}
\section{Lasers for Particle Acceleration}

Explain what a laser stands for, stimulated emission, lasing medium, what the Ti:sapphire crystal does and why it is chosen, CPA

\subsection{Chirped Pulse Amplification}

\subsection{...}

%\section{Glossaries}
%Glossaries are disabled in this version of the template - if you are interested get osudiss-template\_glossary.
%The latex glossaries package can be useful for keeping track of your acronyms, and making a nice hyperlinked 
%list at the beginning of your document.  Note: a glossary/list of acronyms is NOT required by the GS.
%If you do wish to include one, it \textbf{must} appear directly after the lists of figures and tables.
%To use the glossaries package, you must load it in your preamble:
%\begin{verbatim}
%\usepackage[acronym, section=chapter]{glossaries} %load package
%\makeglossaries %required to actually make a glossary
%%A list of common acronyms
%Only those used will be displayed, so you can just add to this list
\newacronym{LLNL}{LLNL}{Lawrence Livermore National Laboratory}
\newacronym{LANL}{LANL}{Los Alamos National Laboratory}
\newacronym{NIF}{NIF}{National Ignition Facility}
\newacronym{NIF-ARC}{NIF-ARC}{National Ignition Facility - Advanced Radiographic Capability}
\newacronym{OMEGA-EP}{OMEGA-EP}{OMEGA Extended Performance System}
\newacronym{RF}{RF}{radio frequency}
\newacronym{QCD}{QCD}{quantum chromodynamics}
\newacronym{QED}{QED}{quantum electrodynamics}
\newacronym{WKB}{WKB}{Wentzel-Kramers-Brillouin}
\newacronym{SRIM}{SRIM}{stopping range of ions in matter}
\newacronym{SOBP}{SOBP}{spread-out Bragg peak}
\newacronym{OSUCCC}{OSUCCC}{The Ohio State Comprehensive Cancer Center}
\newacronym{LLE}{LLE}{Laboratory for Laser Energetics}
\newacronym{TNSA}{TNSA}{target normal sheath acceleration}
\newacronym{IBA}{IBA}{Ion Beam Analysis}
\newacronym{IMPT}{IMPT}{intensity modulated proton therapy}
\newacronym{ALLS}{ALLS}{Advanced Laser Light Source}
\newacronym{XPIF}{XPIF}{x-ray and particle-induced fluorescence}
\newacronym{EDX}{EDX}{energy dispersive x-ray fluorescence}
\newacronym{TEM}{TEM}{transverse electromagnetic mode}
\newacronym{FWHM}{FWHM}{full width at half maximum}
\newacronym{eTNSA}{eTNSA}{enhanced target normal sheath acceleration}
\newacronym{LWFA}{LWFA}{laser wakefield acceleration}
\newacronym{PIC}{PIC}{particle-in-cell}
\newacronym{NGP}{NGP}{nearest grid point}
\newacronym{CCD}{CCD}{charge coupled device}
\newacronym{RPA}{RPA}{radiation pressure acceleration}
\newacronym{LSP}{LSP}{Large Scale Plasma: An implicit particle-in-cell code}
\newacronym{NN}{NN}{neural network model}
\newacronym{SVGP}{SVGP}{stochastic variational gaussian process}
\newacronym{GPR}{GPR}{gaussian process regression}
\newacronym{POLY}{POLY}{polynomial regression}
\newacronym{ML}{ML}{machine learning}
\newacronym{AI}{AI}{artificial intelligence}
\newacronym{HEDS}{HEDS}{high energy density science}
\newacronym{BO}{BO}{bayesian optimization}
\newacronym{WP-ELL}{WP-ELL}{Extreme Light Laboratory at the Wright-Patterson Air Force Base}
\newacronym{GPU}{GPU}{Graphics Processing Unit}
\newacronym{OSC}{OSC}{Ohio Supercomputer Center}
\newacronym{SVR}{SVR}{Support Vector Regression}
\newacronym{RBF}{RBF}{Radial Basis Function}
\newacronym{MAPE}{MAPE}{Mean Absolute Percentage Error}
\newacronym{MSE}{MSE}{Mean Squared Error}
\newacronym{RMSE}{RMSE}{Root Mean Squared Error}
\newacronym{AFIT}{AFIT}{Air Force Institute of Technology}
\newacronym{CSUCI}{CSUCI}{California State University - Channel Islands}
\newacronym{CSU}{CSU}{Colarado State University}
\newacronym{RAL}{RAL}{Rutherford Appleton Laboratory}
\newacronym{DAQ}{DAQ}{Data Acquisition System}
\newacronym{OAP}{OAP}{off-axis parabolic mirror}
\newacronym{HDF5}{HDF5}{Hierarchical Data Format 5}
\newacronym{EPICS}{EPICS}{Experimental Physics and Industrial Control System}
\newacronym{GUI}{GUI}{Graphical User Interface}
 %load list of acronyms contained in acronyms.tex
%\end{verbatim}
%Then, to define an acronym, you will need to include it in your preamble:
%\begin{verbatim}
%\newacronym{AFM}{AFM}{atomic force microscopy}
%\end{verbatim}
%In this template, the acronyms are all placed in a separate file called
%\verb#acronmys.tex# which is then loaded with \verb#%A list of common acronyms
%Only those used will be displayed, so you can just add to this list
\newacronym{LLNL}{LLNL}{Lawrence Livermore National Laboratory}
\newacronym{LANL}{LANL}{Los Alamos National Laboratory}
\newacronym{NIF}{NIF}{National Ignition Facility}
\newacronym{NIF-ARC}{NIF-ARC}{National Ignition Facility - Advanced Radiographic Capability}
\newacronym{OMEGA-EP}{OMEGA-EP}{OMEGA Extended Performance System}
\newacronym{RF}{RF}{radio frequency}
\newacronym{QCD}{QCD}{quantum chromodynamics}
\newacronym{QED}{QED}{quantum electrodynamics}
\newacronym{WKB}{WKB}{Wentzel-Kramers-Brillouin}
\newacronym{SRIM}{SRIM}{stopping range of ions in matter}
\newacronym{SOBP}{SOBP}{spread-out Bragg peak}
\newacronym{OSUCCC}{OSUCCC}{The Ohio State Comprehensive Cancer Center}
\newacronym{LLE}{LLE}{Laboratory for Laser Energetics}
\newacronym{TNSA}{TNSA}{target normal sheath acceleration}
\newacronym{IBA}{IBA}{Ion Beam Analysis}
\newacronym{IMPT}{IMPT}{intensity modulated proton therapy}
\newacronym{ALLS}{ALLS}{Advanced Laser Light Source}
\newacronym{XPIF}{XPIF}{x-ray and particle-induced fluorescence}
\newacronym{EDX}{EDX}{energy dispersive x-ray fluorescence}
\newacronym{TEM}{TEM}{transverse electromagnetic mode}
\newacronym{FWHM}{FWHM}{full width at half maximum}
\newacronym{eTNSA}{eTNSA}{enhanced target normal sheath acceleration}
\newacronym{LWFA}{LWFA}{laser wakefield acceleration}
\newacronym{PIC}{PIC}{particle-in-cell}
\newacronym{NGP}{NGP}{nearest grid point}
\newacronym{CCD}{CCD}{charge coupled device}
\newacronym{RPA}{RPA}{radiation pressure acceleration}
\newacronym{LSP}{LSP}{Large Scale Plasma: An implicit particle-in-cell code}
\newacronym{NN}{NN}{neural network model}
\newacronym{SVGP}{SVGP}{stochastic variational gaussian process}
\newacronym{GPR}{GPR}{gaussian process regression}
\newacronym{POLY}{POLY}{polynomial regression}
\newacronym{ML}{ML}{machine learning}
\newacronym{AI}{AI}{artificial intelligence}
\newacronym{HEDS}{HEDS}{high energy density science}
\newacronym{BO}{BO}{bayesian optimization}
\newacronym{WP-ELL}{WP-ELL}{Extreme Light Laboratory at the Wright-Patterson Air Force Base}
\newacronym{GPU}{GPU}{Graphics Processing Unit}
\newacronym{OSC}{OSC}{Ohio Supercomputer Center}
\newacronym{SVR}{SVR}{Support Vector Regression}
\newacronym{RBF}{RBF}{Radial Basis Function}
\newacronym{MAPE}{MAPE}{Mean Absolute Percentage Error}
\newacronym{MSE}{MSE}{Mean Squared Error}
\newacronym{RMSE}{RMSE}{Root Mean Squared Error}
\newacronym{AFIT}{AFIT}{Air Force Institute of Technology}
\newacronym{CSUCI}{CSUCI}{California State University - Channel Islands}
\newacronym{CSU}{CSU}{Colarado State University}
\newacronym{RAL}{RAL}{Rutherford Appleton Laboratory}
\newacronym{DAQ}{DAQ}{Data Acquisition System}
\newacronym{OAP}{OAP}{off-axis parabolic mirror}
\newacronym{HDF5}{HDF5}{Hierarchical Data Format 5}
\newacronym{EPICS}{EPICS}{Experimental Physics and Industrial Control System}
\newacronym{GUI}{GUI}{Graphical User Interface}
#, to
%make the main source file easier to read.  You can also just put all
%your acronyms directly in the header of your main source file.
%
%Once you have defined an acronym, you can use it with the \verb#\gls{<label>}# command.
%The first time you use the acronym, the full definition is printed: atomic force microscopy (AFM). %\gls{AFM}.
%On subsequent uses, just the abbreviation is printed: AFM %\gls{AFM}
% - Latex keeps track
%for you, so you don't have to do this manually.  There are fancier forms
%of \verb#\gls# that allow you to capitalize (\verb#\Gls{<label>}#) or use the 
%plural (\verb#\glspl{<label>}#) forms of your abbreviations.  See 
%for example \url{http://en.wikibooks.org/wiki/LaTeX/Glossary} for details
%on the glossaries package.

%If you want to list all your acronyms at the beginning of your document, you
%will need to include the \verb#\makeglossaries# command in your preable, as
%shown above, and then 
%\begin{verbatim}
%\printglossary[type=\acronymtype]
%\end{verbatim}
%where you want the list of abbreviations to actually appear 
%(as stated above, this must be right after your lists of figures and tables in the roman
%numeral pages at the beginning of the document).  Then you have
%to go to the terminal (in the directory containing your document), and run the command
%\begin{verbatim}
%makeglossaries (yourdocumentname)
%\end{verbatim}
%to actually create the glossary.  If you just want Latex to keep track of your acronyms for you 
%and print no list, you can just skip this step and the \verb#printglossary# command.

%To make the heading for the List of Abbreviations look the same as the List of Figures
%and List of Tables, instead of using \verb#\printglossary# I defined a custom command to print the glossary:
%\begin{verbatim}
%\newcommand\PrintListofAbbreviations[1]{
%\phantomsection
%\addcontentsline{toc}{front}{\typesetColumnHeading{#1}}
%\printglossary[type=\acronymtype,title={\protect {\typesetLevelTwo{#1}}}]
%%\end{verbatim}
%This command functions the same as \verb#\printglossary#, but instead
%of typesetting the heading like a chapter heading (in smallcaps), 
%it typesets in bold like the lists of figures and tables.  
%To use it, just call \verb#\PrintListofAbbreviations{List of Abbreviations}#
%right after \verb#\listoftables# and before \verb#\mainmatter#.
%If you wish to use a different title for you list of abbreviations,
%just call \verb#\PrintListofAbbreviations{Your Title}#.
%I am unsure if the GS has any restrictions on what title you can use.
%
%One final comment, this template uses \verb#makeindex# to create the glossaries
%for compatability (some versions of Ubuntu for example don't support \verb#xindy#).
%However, there is a better program for making glossaries that you can use by calling
%\begin{verbatim}
%\usepackage[xindy,toc,acronym, section=chapter]{glossaries}
%\end{verbatim}
%instead.  In particular, \verb#xindy# allows you to use symbols in your abbreviations,
%such as Greek letters or accented characters, which are not supported by \verb#makeindex#.
%
%This figure also illustrates the use of subfigures via the subfig package, loaded by placing 
%\verb#\usepackage{subfig}# in the preamble of \verb#template.txt#.  This allows you to include
%each subfigure as an individual image, and have latex arrange and label them for you 
%(including an optional name for each subfigure in \verb#[]#).  This also allows
%you to individually reference Fig.~\ref{one_battery} and Fig.~\ref{five_battery}.
%Note that many journals DO NOT support this, and require that each figure be a single
%image, but this should be fine for a thesis.  
%
%A footnote\footnote{Another foot.}

\section{Applications}

This section highlights some of the most important applications of laser-based proton accelerators that motivates much the projects I have worked in during my PhD studies: proton therapy, 

\acrfull{NIF}

\acrshort{NIF}

\acrfull{LANL}


\acrshort{LANL}[hello]

\subsection{Proton Therapy}

Due to having no cure, cancer is one of the largest medical challenges faced worldwide and generally requires the use of harsh treatments like invasive surgery, chemotherapy, and immunotherapy. Another relevant treatment is called radiotherapy and more than half of all people with cancer will receive it as a part of their medical care \cite{Mayo_2024_Cancer}. Typically, a large machine will provide a source of x-rays (a type of energetic ionizing radiation) that kills the tumor. However, this radiation does not discern whether the cells are cancerous or not -- healthy tissue along the radiation beam path surrounding the tumor will also be damaged. This damage can be mitigated by shooting many beams from different angles such that they overlap at the site of the tumor. In this way, the dose delivered in the beam overlap region will be significantly higher than the surrounding tissue. This approach is typically employed by situating the machine on a rotating ``gantry''

As early as 1905, Bragg \cite{Bragg_1905_JOS} identified that charged particles have different properties than x-rays when passing through matter. Specifically, he identified that radium particles lost more energy (i.e. delivered a higher dose) at a lower speed. Physically, the slower the radium particles, the more time it has to interact with each of the individual atoms and the more energy it can deposit. This means that when the radium first enters a material at its highest speed, it is losing energy slowly in contrast to when it is at a slow speed and about to stop it is losing energy very quickly. In 1946, Wilson \cite{Wilson_1946_Rad} identified that this property of charged particles would enable a more concentrated dose to be delivered closer to the site of a tumor. In \autoref{fig:bragg_curve}, these differences are explicitly highlighted between x-rays and proton beams traveling through water. It can be seen that the proton beam is sharply peaked at a particular distance of around 23 centimeters, where the x-ray beam delivers a relatively higher dose at just a few centimeters. One can see the advantage of protons quite readily from this picture -- if a tumor is located 23 centimeters into the body, the protons, comparatively to the x-rays, will deliver more energy at the tumor site and less energy to the surrounding tissue. The shape of the proton beam curve in this graph is appropriately referred to as a ``Bragg Curve'' which peaks at the ``Bragg Peak''. 

The specifics of the depth-dose curve depend on the material that the particles are traveling through as well as initial energy of the protons. These conditions are well-studied by empirical measurements of the stopping range of ions in matter (SRIM) \cite{Ziegler_2010_SRIM} and can be combined with other techniques to achieve a depth-dose curve that not only peaks at different depths, but also has a spread-out Bragg peak (SOBP) that expands the region where the highest dose is delivered. The most modern form of proton therapy is called intensity-modulated proton therapy (IMPT) whose intensities are modulated to optimally balance tumor dose and sparing of normal tissues \cite{Mohan_2022_PRO}. The first proton therapy center opened at the Loma Linda University Medical Center in California in 1990, but today there are more than 100 centers around the world with more that are planned or in the process of construction. \cite{Mohan_2022_PRO}. 

\begin{figure}
	\centering
	\includegraphics[width=0.75\linewidth]{planning/images/bragg_curve.PNG}
	\caption{The dose delivered as a function of depth traveled in water for two types of beams are depicted -- 200 MeV protons and 16 MV x-rays.}
	\label{fig:bragg_curve}
\end{figure}
 
Right here in Columbus, The Ohio State Comprehensive Cancer Center (OSUCCC), in collaboration with Nationwide Children's Hospital opened a 55,000 square foot proton therapy center in December of 2023 \cite{OSU_CCC} that uses IMPT. Despite all the aformentioned benefits of proton therapy, the intial cost is tremendous -- the OSUCCC was a 100 million dollar investment. The significant cost demonstrates the value this could bring to Central Ohio. The facility is even outfitted with the capability to perform FLASH therapy -- a newer, experimental form of proton therapy that can be delivered in seconds that is still in clinical trials \cite{OSU_CCC} -- which shows the continued investment and innovation in this field. 

The cost, however, cannot be overlooked. The conventional cyclotron accelerators used to accelerate the protons are extremely large and expensive. In recent years, it has been proposed that laser-based particle accelerators could be used to generate high energy protons. These facilities could in principle be smaller and less costly, but the technology is not adequately matured to be considered in the near future \cite{Linz_2016_LaPB}. Laser-based sources are typically only able to generate protons in the 10s of MeV reliably (as opposed to the 100s of MeV required for clinical operation), possess orders of magnitude smaller numbers of protons, and poor reproducibility of the laser pulse output. In addition, the conventional accelerators have already made significant strides in terms of reducing cost, increasing beam quality, and reducing size in recent years \cite{Linz_2016_LaPB} which is something laser-based sources will need to keep up with in the future. However, the potential of developing a smaller and lower cost proton accelerator remains an important motivating factor for many laser-plasma physicists for the coming years.

\subsection{Proton Radiography and Laser Fusion}

All of us are familiar with the use of visible light to image things. A camera's flash will send out a burst of light and the camera will record the light reflected off an object to image it. Other frequencies of light not visible to the naked eye are used all the time to image things as well. High frequency sources like x-rays are used at the dentist due to their ability to probe matter within your body. Radio waves reflect well off of electrically conductive materials like the metals in vehicles which make them ideal for military applications. In an similar way, particles can be used to image objects by analyzing how they interact. Electron microscopes have been used extensively in the past century to image materials at a much higher resolution than an ordinary visible light microscope. From \autoref{fig:bragg_curve}, we know that protons interact with matter in a different way than electromagnetic radiation like x-rays. These differences can be exploited to image things that cannot be done well with other types of radiation. To give one example, a proton is a charged particle that gets deflected by electric and magnetic fields which can enable scientists to obtain information about these fields in a way that x-rays cannot.

One example of proton radiography using laser-accelerated protons is through the National Ignitition Facility's Advanced Radiographic Capability (NIF-ARC). As it's name implies, NIF-ARC is typically used as light source to collect radiograph images of laser fusion experiments at NIF. These fusion experiments involve using a high powered lasers to compress a millimeter sized frozen pellet of hydrogen fuel (specifically, heavier isotopes of hydrogen called deuterium and tritium) to such high temperatures and pressures that the atomic nuclei fuse together into helium. The sun is an example of a high temperature and pressure environment that is able to sustain fusion as its vast source of energy, but we must be a little bit more clever on Earth to obtain conditions like that of in the core of our sun. A similar capability to NIF-ARC exists at the Laboratory for Laser Energetic's (LLE) OMEGA Extended Performance system (OMEGA-EP). The OMEGA-EP provides diagnostics for the main OMEGA laser which performs fusion experiments by directly irradiating a fusion pellet. This is in contrast to the NIF which indirectly compresses a fusion pellet by first irradiating a gold can that surrounds the pellet. These approaches are aptly called indirect and direct drive fusion and are depicted in FIGURE ...

Conventional sources of protons for radiography purposes from linear accelerators like the pRad at Los Alamos National Laboratory (LANL). LANL uses radio frequency (RF) waves of around 200 MHz to accelerate particles up to 800 MeV (SOURCE). The limitation of conventional RF accelerators is the relatively low frequency. If we take the reciprocal of 200 MHz, we find that the separation between individual waves (the period) is around 5 nanoseconds. This may seem like a short time, but on the scale of femtosecond and picosecond ultra-intense pulses, this is a thousand or a million times longer! If one wanted to take proton radiographs of some experiment happening on a picosecond timescale, the conventional accelerator would not work. On the other hand, lasers typically used for laser-plasma experiments are in the infrared and have wavelengths of around 1 micron. A quick calculation tells us that the period is only around 3 femtoseconds. Since the lasers are pulsed at 3 femtoseconds, the emitted protons would also be pulsed at 3 femtoseconds which would theoretically allow us to capture 1 proton radiograph every 3 femtoseconds. Another advantage of these laser-accelerated proton beams is that they are emitted from a very small spot, on the order of microns, which is beneficial for obtaining a higher quality image.

Proton radiography has been demonstrated successfully at the NIF through NIF-ARC \cite{Simpson_2021_PPCF} and at OMEGA through OMEGA-EP \cite{Zylstra_2012_RSI} through proton beams produced from laser-irradiated metallic foils.


\section{This Work}


Talk about prior work 


Talk about motivating question 


Talk about how work is organized. 



\chapter{Physics Background} \label{ch:2}

At the most basic level, a plasma is a state of matter that largely consists of charged particles (rather than neutral particles). Typically, this is achieved by heating a solid, liquid, or gas so much that the constituent atoms achieve a sufficient level of ionization. Francis Chen, author of a prominent plasma physics textbook \cite{Chen_2015_Plasma}, describes a plasma as

\begin{quote}
	a \emph{quasineutral} gas of charged and neutral particles which exhibits \emph{collective behavior}
\end{quote}
In this chapter, laser fields and the physics of charged particles oscillating in them will be overviewed to better understand the plasmas produced from laser-matter interactions. Plasma properties and their relation to quasineutrality and collective behavior will be discussed. Then, ion-acceleration will be explained in the context of laser-plasma interactions.
%\begin{table}
%\begin{center}
%\begin{tabular}{lll}
%\hline
%\\[3pt]
%Alphabet Character & Vowel & Number \\[3pt]
%\hline
%\\[3pt]
%A & Yes & 1 \\
%B & No & 2 \\
%C & No & 3 \\[3pt]
%\hline
%\end{tabular}
%\label{tbl3}
%\caption{Another Table}
%\end{center}
%\end{table}

\section{Electric Fields}


\subsection{Gaussian Laser}
In order to heat up a material with a laser efficiently, the energy would ideally be concentrated to a small point. Lasers are a coherent source of light that can be focused to narrow beams. The intended output of many lasers is the fundamental transverse electromagnetic mode \cite{Zangwill_2012} (TEM$_{00}$) described by the following electric field

\begin{equation}
	E(r, x) = E_0 \hat{y} \frac{w_0}{w(x)} \exp( - \frac{r^2}{w(x)^2}) \cos(k x - \arctan(\frac{x}{x_R}) + \frac{k r^2}{2 R(x)}) \label{eq:gaussian_beam}
\end{equation}
where $\hat{x}$ is the propagation direction, $\hat{y}$ is the polarization direction, and $r = \sqrt{y^2 + z^2}$ is the radial distance away from the laser axis. The radius of curvature $R(x)$ is inversely related to how strongly the wavefronts are curved and is infinity at $x=0$ and minimum at $x=x_R$. The guoy phase $\arctan(x/x_R)$ is an observed phase of $\pi$ that is continuously picked up as the beam propagates from $x=-\infty \rightarrow x=+\infty$. Additionally, the beam radius $w$ is expressed as

\begin{equation}
	w(z) = w_0 \sqrt{1 + (x/x_R)^2} \label{eq:beam_radius}
\end{equation}  
and has a minimum value at the \emph{beam waist} $w(0) \equiv w_0$ at the focal position of the laser. The length scale over which the beam can propagate without diverging significantly is the \emph{Rayleigh range} $x_R \equiv \frac{\pi w_0^2}{\lambda}$. A beam with these properties is usually referred to as a gaussian beam\footnote{I've googled (actually bing'd unfortunately) more times than I can count ``gaussian beam wikipedia'' over the course of my studies. I don't know why I didn't bookmark that page.} and is depicted in \autoref{fig:gaussian_beam}.

\begin{figure}
	\centering
	\includegraphics[width=0.75\linewidth]{planning/images/gaussian_beam.png}
	\caption{The intensity profile of a gaussian beam with laser parameters seen in the Extreme Light Laboratory at Wright-Patterson Air Force Base ($\lambda = 0.8 \mu$m, $I_0=\SI{5e18}{\watt \per \centi \meter \squared}$, $w_0 = \SI{1.5}{\micro \meter}$, $\tau_\text{FWHM}=40$fs). The beam radii $w_0$ and $w(x_R)$ are displayed in addition to the rayleigh range $x_R$.}
	\label{fig:gaussian_beam}
\end{figure}

The peak intensity is related to the electric field by $I_0 = \frac{1}{2} \epsilon_0 c E_0^2$ and \cref{eq:gaussian_beam} shows that the intensity decays as $I(x, r) = I_0 (w_0 / w(x))^2 \exp(-2 r^2 / w(x)^2)$ with increasing $r$ and $x$. If we integrate this intensity distribution over the entire $y-z$ plane (at $x=0$), we obtain the peak power $P_0 = \frac{\pi \omega_0^2}{2} I_0$. Furthermore, we can integrate the power over the pulse duration (assuming the pulse has a $sin^2(t)$ envelope) to obtain the total energy in the pulse 

\begin{equation}
	E = \frac{\pi \omega_0^2}{2} I_0 \tau_\text{fwhm} \label{eq:gaussian_beam_energy}
\end{equation}
where $\tau_\text{fwhm}$ is the full-width half-max and is equal to half the pulse duration for a $sin^2(t)$ pulse envelope. 

\subsection{Single Particle Motions}

Electrostatics is governed by the Maxwell's equations which describe the allowed wave-like solutions for electric and magnetic fields in both matter and vacuum. The most relevant equation in this section is Gauss' Law or Poisson's Equation which can be expressed as 

\begin{equation}
	- \nabla^2 \phi = \nabla \cdot \vec{E} \equiv \frac{\rho}{\epsilon_0} \label{eq:poisson}
\end{equation}
which relates the electrostatic potential $\phi$ or electric field $E$ to the charge density $\rho$. This equation highlights how electric fields are directed radially outward from positive charges and inward towards negative charges. The motion of an electron in the influence of an electric field $E$ or magnetic field $B$ is given by the \emph{Lorentz force} $F_L$

\begin{equation}
	F_L \equiv -e (\vec{E} + \vec{v} \times \vec{B}) \label{eq:lorentz}
\end{equation}

\subsubsection{Quiver Energy}
To gain intuition about some quantities of interest for laser-matter interactions, let's consider a simple problem of an electron of charge $-e$ governed by \cref{eq:lorentz} with a negligible magnetic field $B$. Additionally, only consider 1D motion in the oscillating field $E = E_0 \cos(\omega t)$ for a laser field of frequency $\omega$. Then, the equation of motion is

\begin{equation}
	\frac{d v}{d t} = - \frac{e E_0}{m} \cos(\omega t)
\end{equation}
We can integrate this equation to obtain the velocity and position as a function of time (assuming $x_0 = v_0 = 0$)

\begin{align}
	v(t) &= - v_\text{osc} \sin(\omega t) \label{eq:v(t)} \\
	x(t) &= \frac{v_\text{osc}}{\omega} [\cos(\omega t) - 1] \label{eq:x(t)}
\end{align}
where $v_\text{osc} \equiv (e E_0) / (m \omega)$ is defined as the \emph{quiver velocity}. From \autoref{eq:v(t)}, we can calculate the kinetic energy gained by an electron as $U_p \equiv \frac{1}{2} m \langle v^2 \rangle = \frac{1}{4} m v_\text{osc}^2$ which is known as the \emph{ponderomotive potential (energy)}. This energy represents the cycle-averaged quiver energy of an electron in an electromagnetic field. A more commonly used term is the dimensionless \emph{normalized vector potential} $a_0$ which is closely related to the quiver velocity

\begin{equation}
	a_0 \equiv v_\text{osc} / c = \frac{e E_0}{m \omega c} \label{eq:a0}
\end{equation}
Ultra-intense laser-matter interactions involve relativistic electrons which are produced when $a_0 \gtrsim 1$ (for a $\lambda \approx \SI{1}{\micro \meter}$ wavelength). In terms of the field, $a_0 \sim 1$ corresponds to a peak electric field $E_0 = \frac{2 \pi m c^2}{e \lambda} \simeq \SI{4}{\tera \volt \per \meter}$. In terms of the peak intensity, $I_0 = \frac{1}{2} c \epsilon_0 E_0^2 \simeq \SI{2e18}{\watt \per \centi \meter \squared}$ for this electric field. Consequently, the threshold for relativistic interactions is commonly understood as $I_0 \gtrsim \SI{1e18}{\watt \per \centi \meter \squared}$. 

\subsubsection{Ponderomotive Force}
The above approach yields some important scales for laser-matter interactions, but only describes the interaction of a plane wave that is spatially homogeneous. A real laser field would be spatially inhomogeneous and we can express $E(x) \approx E_0 + x E_0'(x)$ to first order. This modifies the equation of motion as 

\begin{equation}
	\frac{d v}{d t} = - \frac{e E_0}{m} \cos(\omega t) - \frac{e E_0'}{m} \cos(\omega t)[ \frac{v_\text{osc}}{\omega}(\cos(\omega t) - 1)]
\end{equation}
where we've inserted the expression for x from \cref{eq:x(t)} which should be approximately true for small $x$. This equation can be simplified and separated into oscillating and non-oscillating components as

% should maybe cite arefiev notes, not sure how, because I can't find it online.
\begin{equation}
	\frac{d v}{d t} = - \frac{e E_0}{m} [\cos(\omega t)(1 - \frac{E_0' v_\text{osc}}{e_0 \omega}) + \frac{E_0' v_\text{osc}}{2 \omega E_0} \cos(2 \omega t)] - \frac{e E_0' v_\text{osc}}{2 m \omega}
\end{equation}
Over many cycles, the oscillating components will average out to zero and the remaining term is given by $\langle F_p \rangle = m \frac{d v}{d t} = - \frac{e E_0' v_\text{osc}}{2 m \omega}$ and is called the \emph{ponderomotive force}. We can generalize this to 3D and express this time-averaged force in several different, equivalent ways

\begin{equation}
	\langle F_p \rangle = - \frac{e^2}{2 m \omega^2} \lvert E_0 \rvert \nabla E_0 = - \frac{m c^2}{4} \nabla(a_0^2) = - \nabla U_p \label{eq:pond_force}
\end{equation}
where 
\begin{equation}
	U_p = \frac{e^2 E_0^2}{4 m \omega^2} = \frac{1}{4} m v_\text{osc}^2 \label{eq:pond_potential}
\end{equation}
is the ponderomotive potential energy introduced earlier. The ponderomotive force is an important mechanism in the absorption of laser energy by electrons which will be expanded upon in \cref{sec:absorption}.

\section{Plasma Physics}
The quasi-neutrality condition reflects the fact that a plasma is charge neutral throughout its volume in a similar way to an ideal conductor: mobile electrons will reorganize themselves in the presence of an external electric field to maintain zero field (or constant  potential). The simplest plasma description will assume the ions are immobile (due to being much heavier than the electrons) and can be treated as a constant neutralizing background density $n_i$ for the electrons of density $n_{e,0} = Z n_i$ (for a plasma with atomic number $Z$).

\begin{figure}
	\centering
	\includegraphics[width=0.75\linewidth]{planning/images/plasma_oscillation.PNG}
	\caption{An initially charge-neutral plasma is depicted on the left. On the right, the electrons are displaced by a distance $x$ creating a charge separation and electric field akin to a parallel-plate capacitor directed towards the right. Adapted from Smith \cite{Smith_2020_Thesis}.}
	\label{fig:plasma_oscillation}
\end{figure}

\subsection{Plasma Electron Oscillations}
A simple example can be illustrated by \cref{fig:plasma_oscillation} which shows a sheet of negative charge density $-\sigma = -e n_e x$ displaced to the right a small distance $x$. The region in the bulk of the plasma will experience a force from the parallel plate ``capacitor'' fields directed to the left.

\begin{equation}
	F = m \frac{d^2 x}{d t^2} = - e \frac{e n_e x}{\epsilon_0}
\end{equation}
which has the form of a restoring force that brings the charge imbalance back to the center of the plasma. This oscillatory motion has an associated frequency 

\begin{equation}
	\omega_{p,e} = \sqrt{\frac{n_e e^2}{m_e \epsilon_0}} \label{eq:omegape}
\end{equation}
that gives the timescale for electron motion in the plasma. This characteristic frequency shows why plasmas support collective motion (in opposition to a neutral gas in which collisions between individual particles only happen). To get a feeling for this timescale, let's assume a somewhat typical electron density \SI{1e29}{\per \meter \cubed} in laser-plasma interactions to yield a timescale of $\omega_{p,e}^{-1} \simeq \SI{0.1}{\femto \second}$.

Naturally (without externally imposed forces), these fluctuations in charge would be caused by thermal motions of electrons with a characteristic speed $v_{th}$ 

\begin{equation}
	v_{th} = \sqrt{\frac{k_B T_e}{m}} \label{eq:vthermal}
\end{equation}
Due to the strong restoring force from the charge separation, the electrons can only move a short distance $\lambda_D$, called the Debye length, out of equilibrium in this timescale. We can estimate this length by equating $v_{th} = \lambda_D / t \simeq \lambda_D \omega_{p,e}$ and solve for $\lambda_D$.

\begin{equation}
	\lambda_D = \frac{v_{th}}{\omega_{p,e}} = \sqrt{\frac{\epsilon_0 k_B T_e}{n_e e^2}} \label{eq:debye}
\end{equation} 
Physically, $\lambda_D$ gives a length scale over which the electrostatic force persists in a plasma. Within a distance $\lambda_D$ from some perturbation, charges will feel a force, and outside this distance, the charges will be completely shielded like that of an ideal conductor. 

\subsection{Fluid Model}
This description of a plasma as a sea of electrons with collective motion that allows wave-like motions naturally lends itself toward a fluid model. The first component of this model stems from \cref{eq:lorentz} whose explicit time and space dependence can be expressed through $\frac{d p}{d t} = m (\frac{\partial v}{\partial x} \frac{\partial x}{\partial t} + \frac{\partial v}{\partial t}) = m (\frac{\partial v}{\partial t} + v \frac{\partial v}{\partial x})$ (just considering one spatial dimension for simplicity). The second component of this model is the effect of the pressure gradient from thermal motions. Particles will tend to migrate from areas of higher pressure to lower pressure, where the thermal pressure is typically given by the familiar ideal gas law equation of state $p = n_e k_B T_e$. Consequently, the equation of motion should have a term that is opposite to the pressure gradient direction (i.e. $- \nabla p$). Combining these two components together in a generalized 3D equation results in

\begin{equation}
	m n_e (\frac{\partial \vec{u}}{\partial t} + (\vec{u} \cdot \nabla) \vec{u}) = -e n_e (\vec{E} + \vec{u} \times \vec{B}) - \nabla p \label{eq:fluid}
\end{equation}
where we've changed the single particle velocity $\vec{v}$ to the fluid velocity $\vec{u}$ and multiplied by the electron density $n_e$ to ensure correct units with the pressure gradient term. Then, let's look for a simple radially symmetric solution where the fluid velocity $u = 0$, magnetic field $B$ is negligible, and the temperature is constant (isothermal).

\begin{equation}
	n_e e E = - k_B T_e \frac{\partial n_e}{\partial r}
\end{equation}
and by relating the electric field to the potential $E = - \frac{dV}{dx}$, this equation can be integrated from $n_{e,0} \rightarrow n_e$ and $0 \rightarrow \phi$ to obtain 

\begin{equation}
	n_e = n_{e,0} \exp(\frac{e \phi}{k_B T_e}) \label{eq:boltzmann}
\end{equation}
which is referred to as the \emph{Boltzmann relation} for electrons \cite{Chen_2015_Plasma}. We can get an approximate solution to this equation when the potential $\phi$ is only slightly larger than the equilibrium $\phi=0$, which can be found when $e \phi << k_B T_e$. We can Taylor expand the density to obtain 

\begin{equation}
	n_e \approx n_{e,0}(1 + \frac{e \phi}{k_B T_e})
\end{equation} 
and if we assume a fully ionized plasma with immobile ions of charge $Z$, the density of ions satisfies $n_{e,0} = Z n_i$ and \cref{eq:poisson} becomes 

\begin{equation}
	\epsilon_0 \nabla^2 \phi = - e n_{e,0} + e n_e = e n_{e,0} [1 + \frac{e \phi}{k_B T_e} - 1] = \frac{e^2 n_{e,0} \phi}{k_B T_e}
\end{equation}
This equation admits solutions of an exponentially decaying potential

\begin{equation}
	\phi(r) = \frac{Q}{r} \exp(-\frac{r}{\lambda_{D}}) \label{eq:shielding}
\end{equation}
where $\lambda_D = \sqrt{\frac{\epsilon_0 k_B T_e}{n_{e,0} e^2}}$ is the Debye length from \cref{eq:debye}. A visualization of the decaying potential from \cref{eq:shielding} is shown in \cref{fig:debye}. Looking at the center panel, we can see the fields drop off quickly within a distance $\lambda_D$ in contrast to the left panel's potential that extends much further out in distance $r$. The exponentially decaying potential is a feature of plasmas and highlights the ability of plasma electrons to shield fields in a distance $\lambda_D$. 

\begin{figure}
	\centering 
	\includegraphics[width=\linewidth]{planning/images/debye_length.png}
	\caption{Visualization of the electric potential as a function of radial distance away from a positive point charge at the origin in three scenarios: vacuum (left), plasma (center), ideal conductor (right). Brighter colors show a higher value of $\phi$. In the center panel, the debye length $\lambda_D$ is shown.}
	\label{fig:debye}
\end{figure}

\subsection{Plasma Conditions}
Putting all this together, we can define several conditions that are characteristic of plasmas \cite{Chen_2015_Plasma}. Quasi-neutrality means that the bulk of the plasma is overall charge neutral, with non-neutral regions generally falling within $\lambda_D$ of some charge imbalance. Notable exceptions to quasi-neutrality are charged particle beams in ultraintense laser experiments which can only exist on a timescale shorter than the time it takes for the coulomb repulsion to blow the plasma apart. If $L$ is the length scale of the system in which the plasma resides, we require that $\lambda_D \ll L$. However, this condition is not sufficient because an ideal conductor has $\lambda_D = 0$ but is not a plasma due to the absence of collective behavior. Collective behavior can be enforced by asserting that there are enough electrons $N_D$ within a spherical volume of radius $\lambda_D$. The corresponding equation is $N_D = n_e (\frac{4}{3} \pi \lambda_D^3) \gg 1$. The final condition is that electrostatic interactions should dominate over collisions because the collective behavior (e.g. plasma oscillations) originates from the electrostatic forces. This means that the period of oscillations ($\omega_{p,e}^{-1}$) should be less than the mean time between collisions. 

\section{Absorption of Energy} \label{sec:absorption}

In order for a laser to couple energy to the plasma electrons, some absorption mechanism needs to take place. The most obvious way that electrons can gain energy is through collisions with other energetic electrons and ions. However, the collision frequency is known to get smaller as the temperature goes up \cite{Gibbon_2005_Plasma}, so much so that plasmas can be treated as collisionless for ultra-intense laser experiments. Below, some of the most common known heating mechanisms are summarized.

\subsection{Critical Density}
First, we will look at how the electric field from an oscillating electric field penetrates a plasma. Using \cref{eq:faraday,eq:ampere} combined with the vector identity $\nabla \times (\nabla \times \vec{E}) = \nabla(\nabla \cdot \vec{E}) - \nabla^2 \vec{E}$ \cite{Zangwill_2012}, we can solve for the vector wave equation in terms of only $\vec{E}$. 

\begin{equation}
	(\nabla^2 - \frac{1}{c^2} \frac{\partial^2}{\partial t^2}) \vec{E} = \mu_0 \frac{\partial J}{\partial t} + \nabla(\nabla \cdot \vec{E}) \label{eq:vectorwave}
\end{equation}
We can look for solutions of $\vec{E} = E(x) \cos(\omega t) \hat{x}$ that vary spatially only in the x direction. We are assuming $E(0)$ is the amplitude of the electric field at the boundary between vacuum $x < 0$ and matter $x > 0$ and wish to understand the form of $E(x)$ when $x > 0$. To proceed, we can assume the current density can be related to the drift velocity \cite{Macchi_2013_Plasma} by $J = -n_e e u$ where $u$ is the electron fluid velocity that satisfies \cref{eq:fluid}. Ignoring $B$ and thermal pressure, this relationship becomes 

\begin{equation}
	\frac{\partial \vec{J}}{\partial t} = \frac{n_e e^2}{m} E = \omega_p^2 \epsilon_0 \vec{E}
\end{equation}
using \cref{eq:omegape} which can be combined with \cref{eq:vectorwave} to obtain a differential equation for the electric field

\begin{equation}
	[\nabla^2 + \frac{\omega^2}{c^2}(1 - \frac{\omega_p^2}{\omega^2})] \vec{E} = 0
\end{equation}
By just focusing on the x-dependence, we can simplify this equation to 

\begin{equation}
	\frac{d^2 E}{d x^2} = \frac{1}{l_s^2}  E
\end{equation}
where $l_s^2 \equiv \frac{c^2}{\omega^2 - \omega_p^2}$ defines the \emph{skin depth} $l_s$. In the case where $\omega < \omega_p$, $l_s^2$ is negative and the solution has a sinusoidal dependence

\begin{equation}
	E(x) = E(0) \cos(x / l_s)
\end{equation}
On the other hand, when $\omega > \omega_p$, $l_s^2$ is positive and the solution has an exponential dependence

\begin{equation}
	E(x) = E(0) \exp(-x / l_s)
\end{equation}
This ``evanescent'' behavior when $\omega > \omega_p$ occurs because the electrons cannot respond fast enough to the higher frequency $\omega$. Since the field cannot propagate effectively for $x > 0$, the plasma ends up reflecting a significant portion of the light. Since wavelength (and frequency) is fixed from the laser, we can reformulate this finding in terms of electron density. The critical density $n_c$ is defined as the electron density where $\omega = \omega_{p,e}$. Using \cref{eq:omegape}, this can be expressed as

\begin{equation}
	n_c \equiv \frac{m \epsilon_0}{e^2} \omega^2 \label{eq:criticaldensity}
\end{equation}
When $n_e > n_c$, the plasma is said to be \emph{overdense} and most of the laser light gets reflected. When $n_e < n_c$, the plasma is said to be \emph{underdense}, and the laser light can propagate through the plasma. 

\subsection{Absorption Mechanisms}
A typical Ti:Sapphire laser has a wavelength of $\SI{0.8}{\micro \meter}$ which corresponds to a critical density of $n_c \simeq \SI{1.7e27}{\per \meter \cubed}$. In this work, two materials are of interest: gold and ethylene glycol which have densities of $\SI{19.3}{\gram \per \centi \meter \cubed}$ and $\SI{1.11}{\gram \per \centi \meter \cubed}$ respectively. These mass densities correspond to a number density of electrons $\SI{5.9e28}{\per \meter \cubed}$ and $\SI{1.1e28}{\per \meter \cubed}$ respectively assuming a singly ionized plasma. If the plasmas were multiply ionized, these densities would be even higher. Even though these plasmas are clearly overdense, experiments show energy is able to efficiently couple to the electrons. Consequently, there must exist mechanisms of absorption that are consistent with the fact that most of the laser energy can only be deposited in a small depth $l_s$ into the plasma. 

\subsubsection{Resonance Absorption}

The previous discussion applied to an electric field directed in the x-direction. For gaussian laser beams, the electric field is always perpendicular to the direction of propagation. So, if the laser beam is to be directed toward a target at $x = 0$, normal incidence would imply that the electric field $E_x = 0$. Additionally, an s-polarized beam would have an electric field only in the z direction. As a result, the typical model of a laser beam depositing energy into a plasma involves a p-polarized laser beam traveling obliquely in the $x-y$ plane with some angle of incidence $\theta_i$ measured with respect to the normal direction of the target. Physically, plasma oscillations occur through fluctuations in density which are going to be the strongest in the x direction due to the interface between vacuum and matter.

Kruer \cite{Kruer_2003_Plasma} argues that the reflection of light at oblique incidence occurs at a density $n_e = n_c \cos^2(\theta_i)$ by enforcing momentum conservation of the electric field component in the $y$ direction. Even though we've discussed density profile as an abrupt step: from 0 to $n_e$ from crossing $x = 0$, an actual experiment would see some pre-heating of the target before the target becomes ionized and behaves as a mirror at the critical density. This can be characterized by some scale length $L_p \equiv n_e (\frac{\partial n_e}{\partial x})^{-1}$ which is smaller for steeper density profiles. As a result, at higher incidence angles, the evanescent portion of the electric field has to travel further into the underdense region of the target to reach the critical density. 

When the frequency of the laser $\omega \simeq \omega_p$, the laser light is in resonance with the plasma oscillations and the energy can be most efficiently absorbed. To maximize the amount of energy reaching the critical density surface in resonance, we would want $\theta_i$ to be large so that the electric field has a significant component in the $x$ direction, but also small enough so that the field doesn't diminish too much by traveling in the evanescent underdense region. Denisov \cite{Denisov_1957_JETP} and others \cite{Forslund_1975_PRA, Freidberg_1972_PRL, Estabrook_1975_PoF} addresses this question and develops a model for this so-called \emph{resonance absorption} An approximate version of this formula is given by Kruer \cite{Kruer_2003_Plasma}

\begin{equation}
	\phi(\tau) \simeq 2.3 \tau \exp(-2 \tau^3 / 3)
\end{equation}
where $\tau \equiv (k L_p)^{1/3} \sin(\theta_i)$ takes into account both the scale length and incidence angle. \cref{fig:absorption} shows the fractional absorption $\phi(\tau)^2/2$ of the incident light from this model as a function of $\theta_i$ for various scale lengths.There is an optimal angle $\theta_{max} \approx \arcsin(0.8 (k L_p)^{-1/3})$ that maximizes the absorption fraction (although Kruer notes that his simple model overestimates the peak absorption -- the  fraction should peak at around 0.5 \cite{Kruer_2003_Plasma}). 

\begin{figure}
	\centering 
	\includegraphics[width=\linewidth]{planning/images/absorption.png}
	\caption{Absorption fraction as a function of incidence angle $\theta_i$. For resonance absorption, the density scale length $L_p$ is varied in terms of the laser wavelength $\lambda = \SI{0.8}{\mu \meter}$. For the Brunel mechanism, fractions are plotted for two regimes $a_0 \ll 1$ (where a value of $a_0 = 0.1$ was chosen) and for $a_0 \gg 1$ (which has no dependence on $a_0$).}
	\label{fig:absorption}
\end{figure}

\subsubsection{Brunel Heating}
Resonance absorption only makes sense when the amplitude of the plasma oscillations $x_\text{osc} = v_\text{osc} / \omega = \frac{a_0 \lambda}{2 \pi}$ is less than the scale length $L_p$ \cite{Gibbon_2005_Plasma}, otherwise there is not enough available space for the oscillations to take place. For $\lambda = \SI{0.8}{\mu \meter}$ and $a_0 = 1$, $x_\text{osc} = \SI{127}{\nano \meter}$. However, for extremely small scale-lengths, efficient electron heating can still be observed \cite{Grimes_1999_PRL}. Consequently, a different type of heating mechanism is responsible in this case (somewhat confusingly) called ``not so resonant, resonant absorption'' \cite{Brunel_1987_PRL}. This model, developed by Brunel, is also known as \emph{vacuum heating} and will be explained below. 

Before explaining the Brunel mechanism, I should the basics of the 3-step model used for high-harmonic generation \cite{Corkum_1993_PRL}. Readers may find this model more familiar due to the recent 2023 Nobel Physics Prize won by Ohio State's Pierre Agostini \cite{Nobel_2023} which utilized the ideas of this model to produce attosecond pulses. This model involves a strong oscillating electric field $E(x) = E_0 \cos(\omega t)$ incident on an atom. Now, assume an electron is ionized at $t = t^*$. Under the influence of the oscillating field, the (initially stationary) electron will gain  and lose energy by moving away from the atom and returning back toward the atom. When $t^* \neq n \pi$ for integer $n$, it is actually possible for the electron to return back to the atom with non-zero energy. In fact, when $\omega t^* \approx 17^\circ + n (180^\circ)$, the electron returns with an energy peaking at $3.17 U_p$ where $U_p$ is the ponderomotive potential given by \cref{eq:pond_potential}. Furthermore, modeling the ionization rate through quantum-mechanical tunneling of an electron through a Coulomb potential warped by the oscillating laser field, we also determine that the most probable energy for an electron is strongly peaked at the $3.17 U_p$ cutoff. In short, this model shows how a laser field can produce electrons with energy on the scale of the ponderomotive potential with high probability at a frequency of twice per optical cycle. 

In the Brunel mechanism \cite{Brunel_1987_PRL}, we are considering a laser field incident on a planar target at $x > 0$ and vacuum at $x < 0$. In order for the electrons to escape the target, there needs to be some component of the electric field in the $x$ direction. Thus, we need to consider oblique incidence and p-polarization just like with resonance absorption. When the plasma scale length is small, the electrons will be able to travel far enough in the $x < 0$ to escape the plasma entirely and gain energy on the order of $U_p$ in a similar fashion to Corkum's model \cite{Corkum_1993_PRL}. Electrons arriving back to the target at just the right time will penetrate deeper than the skin depth $l_s \approx c / \omega_p$ and be inaccessible to the laser field \cite{Gibbon_2005_Plasma}. These \emph{hot electrons}, generated primarily on the front surface of the target, will provide the energy to heat the remainder of the overdense region of the target that the laser field cannot directly access. 

As mentioned, the optimal angle would appear to be for grazing incidence ($\theta_i = 90^\circ$), but Gibbon notes that accounting for imperfect reflection of the laser field and relativistic energies of the electrons, the efficiency no longer diverges at $\theta = 90^\circ$ \cite{Gibbon_2005_Plasma}. In \cref{fig:absorption}, some estimates for the absorption efficiency are plotted according to a simplified model developed by Gibbon \cite{Gibbon_2005_Plasma} based on Brunel \cite{Brunel_1987_PRL}.

\subsubsection{Other Mechanisms}
When the laser field penetrates a distance $l_s \approx c / \omega_p$ into the overdense region of the target, the electrons can heat up through collisions at an absorption rate $\eta \propto \frac{\nu_{ei}}{\omega_p,i}$ \cite{Gibbon_2005_Plasma} where $\nu_{ei}$ is the electron-ion collision frequency. This type of absorption is called the \emph{skin effect}. In this case, we see Fresnel-like reflection and absorption (see \cite{Griffiths_2017}) that is effective for large incidence angles. Even when the collision frequency is low, we can still get efficient absorption as long as the thermal electron motions are large compared to the skin depth (i.e. $v_{th}/\omega > l_s$) \cite{Gibbon_2005_Plasma}. This phenomena is called the \emph{anomalous skin effect} that is also most effective at large incidence angles. 

All of the mentioned phenomena work best at oblique incidence in p-polarization. But, for relativistic intensities, additional heating mechanisms arise. When $a_0 \gtrapprox 1$, the magnetic portion of \cref{eq:lorentz} becomes significant. At normal incidence, the electric and magnetic field components both fall in the $y-z$ plane. The electric fields move the electrons strongly causing a current $\vec{J}$ which in turn will interact with the laser magnetic field $\vec{B}$ in the direction perpendicular to both. As a result, this type of heating is known as $\vec{J} \times \vec{B}$ heating \cite{Gibbon_2005_Plasma,Kruer_1985_PoF}. Because $\vec{J} \times \vec{B}$ is in the direction of propagation, this effect is most pronounced at normal incidence. (CAN Maybe talk about how circular polarization mitigates JxB heating)

At even higher intensities, the laser can directly impart energy to the electrons through radiation pressure \cite{Macchi_2013_RevModPhys} because photons themselves carry momentum. These mechanisms are explored further in \cref{sec:acceleration}. In reality, all experiments involve a combination of several different electron heating mechanisms. Consequently, many experiments and simulations have been devoted to parametric studies that show how parameters ($L_p$, $\theta_i$, polarization, $a_0$, etc.) affect electron absorption (CITATIONs needed)

\section{Ion Acceleration} \label{sec:acceleration}

The previous section gave an overview of the laser-plasma interactions and how they can efficiently couple energy into hot electrons. Regardless of heating mechanism, one theme is common to all -- the energy gained by an electron's quiver motion in an oscillatory field, known as the ponderomotive potential (\cref{eq:pond_potential}) sets the scale for the hot electron temperature $T_h$. That equation was only valid for non-relativistic electrons, so we must replace $U_p = \frac{1}{4} m v_\text{osc}^2$ with the relativistic kinetic energy $U_p \equiv (\gamma - 1) m c^2$ where $\gamma = 1 / \sqrt{1 - \frac{v_\text{osc}^2}{c^2}}$ defines the lorentz factor. We can combine this with the relativistic momentum $p = \gamma m v_\text{osc}$ and \cref{eq:a0} to determine an approximate expression of $\gamma$ in terms of $a_0$
	
\begin{equation}
	\gamma \simeq \sqrt{1 + a_0^2} = \sqrt{1 + \frac{I_{18} \lambda_{\mu m}^2}{1.37}} \label{eq:gamma}
\end{equation}
where $I_{18}$ is the peak intensity of the laser pulse in $\SI{1e18}{\watt \per \centi \meter \squared}$ and $\lambda_{\mu m}$ is the wavelength in $\mu m$. In 1992, Wilks \cite{Wilks_1992_PRL} conducted simulations to show that $T_h$ is on the order of $U_p$. 

\begin{equation}
	k_B T_h = m c^2 (\gamma - 1) \label{eq:wilks}
\end{equation}
In \cref{fig:electron_scaling}, we can see that the \emph{Wilks scaling} (pink) closely matches ultra-intense laser experiments. The other scalings in the figure are similarly validated by computational simulations and are all proportional to $(I \lambda^2)^\alpha$, where $0 < \alpha \leq 1$. As a result, the product $I \lambda^2$ is an important quantity in laser-plasma experiments and is called the \emph{irradiance}. Wilks also outlines a way to measure the hot electron temperature in his simulations \cite{Wilks_1992_PRL} by taking the slope of $\frac{dN_e}{dE}$ in the $MeV$ regime and is something I do in (MAYBE REFERENCE FIGURE FROM CH4 of HOT ELECTRON SPECTRA)

\begin{figure}
	\centering 
	\includegraphics[width=0.75\linewidth]{planning/images/gibbon_hot_electron.PNG}
	\caption{Experimentally recorded hot electron temperatures as a function of irradiance $I \lambda^2$ are plotted as red squares. The empirical scaling models are given by Wilks \cite{Wilks_1992_PRL}(pink, solid), Gibbon and Bell \cite{Gibbon_1992_PRL}(blue, ashed), Forslund et. al. \cite{Forslund_1977_PRL}(green, dash-dot), and Brunel \cite{Brunel_1987_PRL}(black, dotted). Figure is taken from Gibbon \cite{Gibbon_2005_Plasma}}
	\label{fig:electron_scaling}
\end{figure}

Since protons are 1836 times as massive as electrons, they are much harder to accelerate and on the scale of femtosecond pulse interactions, they are essentially immobile. Despite this, ultra-intense laser experiments have demonstrated proton acceleration is possible. This section will explain the \gls{TNSA} mechanism for accelerating protons and light ions which is heavily dependent on $T_h$. Then, we will discuss alternative acceleration mechanisms. Finally, we'll overview some important applications.

\subsection{Target Normal Sheath Acceleration}
The observation of energetic protons off the rear side of thin plastic and gold targets has been documented throughout a variety of experiments since the 80s \cite{Tan_1984_PoF}. It might sound unintuitive that we would even see protons in the first place; after all, when shooting a target like aluminum, one would expect aluminum ions. It turns out that there is always an important and measurable surface contamination layer, primarily composed of hydrogen and light hydrocarbons \cite{Gitomer_1986_PoF}. Allen \cite{Allen_2004_PRL} showed that when removing the surface contaminant from the backside, we see a strong suppression in ion acceleration. This points to the contaminant layer being the crux of what is accelerated.

\subsubsection{TNSA Models}

Expansion models have been long known since the 70s and 80s (e.g. Crowe \cite{Crow_1975_JPP} and Kishimoto \cite{Kishimoto_1983_PoF}) that describe the acceleration of protons with experiments (e.g. Tan \cite{Tan_1984_PoF}) as well. However, the advent of Chirped Pulse Amplification \cite{Strickland_1985_Optics} in 1985 by Donna Strickland and Gerard Mourou allowed the intensities of the laser light to increase to relativistic levels ($a_0 > 1$) with sub-ps pulse durations. This technology dramatically impacted the field of laser-plasma interactions because it allowed new relativistic regimes of ion acceleration to be explored -- for this work they won the Nobel Prize \cite{Nobel_2018}. 

The field of ultra-intense proton acceleration kicked off in the year of 2000 with a group at Michigan \cite{Maksimchuk_2000_PRL} finding 1.6 MeV protons from a thin aluminum foil with a $\SI{3e18}{\watt \per \centi \meter \squared}$ class laser at normal incidence. Then, Rutherford Appleton Laboratory found 30 MeV protons \cite{Clark_2000_PRL} from a $\SI{5e19}{\watt \per \centi \meter \squared}$ class laser incident on a lead target at $45^\circ$ incidence. Shortly after, \gls{LLNL} found energies up to 58MeV \cite{Snavely_2000_PRL} from a $\SI{3e20}{\watt \per \centi \meter \squared}$ class laser on a gold target at $45^\circ$ incidence.

Now that efficient MeV proton acceleration had been achieved through multiple studies, a more thorough comprehensive picture of the physical process was desired. In 2001, Wilks \cite{Wilks_2001_PoP} summarized much of the existing literature including the isothermal expansion model \cite{Crow_1975_JPP}, existence of a maximum cutoff energy \cite{Kishimoto_1983_PoF}, and dependence on hot temperature \cite{Wilks_1992_PRL}. Then, he described the \gls{TNSA} process in the following way \cite{Wilks_2001_PoP}

\begin{quote}
	... the prepulse creates large plasma in front of a solid target. Once the main pulse hits the target, a cloud of energetic electrons (1-10 MeV in effective temperature) is generated, which extends past the ions on both the front and back of the target. Since the protons on the back are in a sharp, flat density gradient, they are accelerated quickly (in the first few $\mu m$ off the target) to high energies in the forward direction ... On the front, the outermost ions are in a sphere, in a long scale length plasma (due to prepulse) and therefore are accelerated to lower energies, and are spread out into $2 \pi$ steradians.
\end{quote}

\begin{figure}
	\centering 
	\includegraphics[width=\linewidth]{planning/images/tnsa.PNG}
	\caption{The \gls{TNSA} process is depicted. First, an intense laser pulse irradiates the front side of a target foil of few $\mu m$ thickness. This generates hot electrons that stream through the foil and re-emerge in a cloud on the rear side. The charge separation of the hot electrons and positively charged target creates intense longitudinal fields ($\sim$ TV/m) that accelerate light ions in the mostly target normal direction. This figure was taken from Roth \cite{Roth_2016_CERN_TNSA}}
	\label{fig:tnsa}
\end{figure}

A visual of the \gls{TNSA} process can be seen in \cref{fig:tnsa}. Although Wilks \cite{Wilks_2001_PoP} provided a physical picture of the \gls{TNSA} process, existing models didn't always match up to experiments. For this reason, the ensuing decade saw much progress in the development of models to describe the spectrum of \gls{TNSA} accelerated protons. Perego \cite{Perego_2011_NIaMiP} gives a good review of some of the leading models developed and tested against experiments in the 2000s and these models will be summarized below. 

First, are the isothermal expansion (fluid) models which include Mora's ``Plasma Expansion into a Vacuum'' \cite{Mora_2003_PRL} (2003) that combine \cref{eq:poisson,eq:boltzmann} with fluid \cref{eq:continuity,eq:lorentz_mora}. This model underlies the work done in \cref{ch:5}, where it is explained in more detail, and has the issue of predicting proton energies that can go up to arbitrarily high values. As a remedy, Mora introduces a finite acceleration time $\tau$ which is of the order of the pulse duration. Mora \cite{Mora_2005_PRE} addresses this in a different way (2005) by instead assuming an adiabatic model and limiting the target to be a thin foil (instead of a semi-infinite target). 

Alternatively, Passoni and Lontano \cite{Passoni_2004_LaPB} introduces an upper limit to the integration range of the electric potential instead of using the fluid equations. In this approach, the electrostatic fields determined from the potential are considered static, and the ensuing ion dynamics is determined by placing a test ion in the field. Further iterations incorporate some distribution of speeds for the electrons (non-relativistic Maxwell-Boltzmann \cite{Lontano_2006_PoP} or relativistic Maxwell-Juttner \cite{Passoni_2008_PRL}) and use an empirically determined scaling for the peak energy of electrons (as a function of laser energy) that do not escape the system \cite{Perego_2012_RevSci}.

Furthermore, some hybrid models include elements of both fluid and quasistatic models like Robinson \cite{Robinson_2006_PRL} and Albright \cite{Albright_2006_PRL}.

\subsubsection{Optimization of TNSA process} \label{sec:tnsa_opt}

Since the \gls{TNSA} process is intimately related to the hot electron process at the front of the target and the flat density gradient at the back, many efforts have been taken to design targets that optimize ion acceleration. Patel \cite{Patel_2003_PRL} used spherically shaped targets to act as a lens that can focused the proton beam. MacKinnon \cite{Mackinnon_2002_PRL} showed lower target thickness leads to higher proton energy due to a higher mean density of hot electrons at the surface. More recent experiments have even used nanowires \cite{Vallieres_2021_Nature} and microtubes \cite{Strehlow_2022_Nature}. Many experiments generally find that there is an optimal level of pre-expansion of the target that enhances hot electron generation and ion acceleration (e.g. McKenna \cite{McKenna_2008_LaPB}).

Another way to increase the peak proton energy of the emitted spectrum is to use two spatially aligned pulses. If one pulse has a delay with respect to the other, the first pulse could pre-expand the target to provide an optimal electron density at the front surface \cite{Ferri_2018_PoP}. If the pulses are also temporally aligned, the constructive interfere at the target front surface may prove beneficial \cite{Ferri_2019_Nat_Comm}. The second approach is called \gls{eTNSA} and \cref{ch:4} is devoted to this phenomenon. 

See the review article by Roth \cite{Roth_2016_CERN_TNSA} for a more comprehensive list of the different approaches to enhance TNSA.

\subsection{Other Acceleration Mechanisms}
\gls{TNSA} is not the only method in which protons can be accelerated. For intensities of greater than $10^{21} \text{ W cm}^{-2}$, laser-induced ion shocks can start to play a significant role \cite{Fuchs_2005_Nat}. For even higher intensities $\sim 10^{23} \text{ W cm}^{-2}$, the radiation pressure of the electromagnetic wave can efficiently transfer momentum to the ions \cite{Fuchs_2005_Nat}. See Macchi \cite{Macchi_2013_RevModPhys} for a more in depth discussion on these topics. One way to differentiate the \gls{TNSA} regime from other regimes is through the following equation relating $a_0$ to various properties of the laser and target

\begin{equation}
	a_0 = n_e \lambda r_e l_0 = 224 \left(\frac{n_e}{\SI{1e29}{\per \meter \cubed}}\right) \left(\frac{l_0}{\SI{1}{\micro \meter}} \right) \label{eq:acc_regime}
\end{equation}

Lezhnin \cite{Lezhnin_2015_PoP} uses this equation to differentiate \gls{TNSA} from two other mechanisms: \gls{RPA} and Coulomb Explosions; this can be seen in \cref{fig:regimes}. If the laser intensity is sufficiently high and density is low enough to be transparent, the laser can quickly sweep away most electrons to leave behind a strongly positive target. The repelling coulomb force will cause the protons to expand outwards in all directions.

When the radiation pressure $P_\text{rad} \approx 2 I_0 / c$ is significant enough to overcome the thermal expanding pressure $n_e k_B T_e$, ions can accelerate directly through the transfer of momentum. \cite{Macchi_2013_RevModPhys}. In this regime, laser absorption into hot electrons by traditional mechanisms would be detrimental. By shooting the laser at normal incidence with circular polarization, resonance absorption and $\vec{J} \times \vec{B}$ heating can be minimized as seen in \cref{sec:absorption}.

For thick targets, this immense pressure can impart a parabolic deformation that allows the laser to penetrate further. This is the regime of \emph{hole boring}. Targets thin enough where the hole boring process reaches the target rear in a time less than the pulse duration are in the \emph{light sail} regime. \cite{Macchi_2013_RevModPhys}. 

\begin{figure}
	\centering 
	\includegraphics[width=0.75\linewidth]{planning/images/acceleration_regimes.PNG}
	\caption{The regimes of various three different acceleration mechanisms are displayed in terms \cref{eq:acc_regime}. This figure was taken from Roth \cite{Lezhnin_2015_PoP}}
	\label{fig:regimes}
\end{figure}

\subsubsection{Wakefield Acceleration}
All the aformentioned acceleration processes only make sense for overdense targets whose electron density is greater than the critical density ($n_e > n_c$). This is because the critical density surface is the primary area where the laser deposits energy into hot electrons. If the target plasma has $n_e < n_c$, the target is said to be underdense and there is no critical density surface where the laser interacts with. Tajima and Dawson \cite{Tajima_1979_PRL} first proposed the idea of a ``Laser Electron Accelerator'' in 1979 that is capable of accelerating electrons to high energies through the non-linear ponderomotive force. If the conditions are just right, the electrons can ``surf'' a plasma wave in the wake of the pulse and pull along positive ions in a process now known as \gls{LWFA}. A comprehensive review of the subject can be found here \cite{Esarey_2009_LPA}.

%\subsection{Notes}
%
%should talk about Bragg Peak and how that works based on coulomb collision
%
%E = kT / e L where L is related to the debye length or scale length
%
%Heavy ions stay still, hot electrons expand out several debye lengths: causes charge separation that accelerates light ions/protons. Characteristic: accelerated in the normal direction even though laser is at oblique incidence. High scale reduces TNSA electric field b/c boundaries are not as sharp on the backside (to make a capacitor like field). 
%
%Spectrum of TNSA beams is typically broadband up to a cutoff energy. The dN/dE looks kind of thermal i.e. exponentially decaying.  
%
%Also, a sharp angular boundary in the proton angular distribution (clearer in higher Z and thicker targets) is consistent with a bell-shaped transverse distribution of hot electrons in the rear surface sheath due to the fact that the density will naturally be higher along the laser axis and decreases with transverse radius. 
%
%Then, they talk about TNSA modeling which I've already covered: Passoni, Mora, etc.
%
%Can look at Joe Smith's TNSA experiment paper (is it on arxiv?) to see better evidence of the relationships and how they fare with experiments.
%
%Can give a simple estimate of the regime it is relevant by considering intensities where we get relativistic ions. However, this is really, really high and thus why we consider TNSA as the dominant mechanism. We don't have lasers this high of an intensity for these affects to be relevant, so it must come from TNSA i.e. strong electrostatic fields from electrons rather than direct electron acceleration.
%
%Then, in the relativistic transparency regime of thin targets, the laser pulse can actually deposit well into the target. Of course, this means that the laser isn't pushing on the front edge of the target so that we aren't in the RPA regime anymore. In this regime, the Breakout afterburner (BOA) model is typically employed. 
%
%Finally collisionless shock acceleration (CSA) becomes relevant when shocks appear and ions reflect off this shock front with speed $2 v_s$ where $v_s = M c_s$ and $M > 1$ is the Mach number. \citep{Fiuza_2013_PoP} talks about how shock acceleration is optimal for near critical density targets that particularly have an exponential scale length plasma on the rear side of $L_g \approx \lambda_0 / 2 \sqrt{m_i/m_e}$ for uniform electron heating and ion reflection. They also say that the target should be thin enough so that $L < 2 \pi c / \omega_{p,i}$ but not too much smaller. 

\chapter{Computational Methods} \label{ch:3}

\section{The Particle-In-Cell Method}

The Particle-In-Cell (PIC) method involves solving Maxwell's Equations on a grid 

\begin{align}
	\nabla \cdot \vec{E} &= \frac{\rho}{\epsilon_0}  \label{eq:gauss} \\
	\nabla \times \vec{E} &= - \frac{\partial \vec{B}}{\partial t} \label{eq:faraday} \\
	\nabla \cdot \vec{B} &= 0 \label{eq:gauss_magnetism} \\
	\nabla \times \vec{B} &= \mu_0 (\vec{J} + \epsilon_0 \frac{\partial \vec{E}}{\partial t}) \label{eq:ampere}
\end{align}
This is combined with the lorentz force

\begin{equation}
	\vec{F} = q(\vec{E} + \vec{v} \times \vec{B}) \label{eq:lorentz_pic}
\end{equation}
which determines the motions (i.e. $\vec{r}$ and $\vec{v}$) of charged particles by integration. It is impossible to keep track of the true numbers of particles in this type of simulation which would be roughly on the order of Avogadro's number $\sim 10^{23}$. Instead, we lump many particles together into what is called a \emph{macro particle}. For example, one ``macro electron'' could contain 1 trillion ``real electrons''. Also, we cannot hope to have infinite precision in calculating quantities of interest. Spatially, we must separate the simulation volume into a grid where each cell has length $\Delta x$, $\Delta y$, and $\Delta z$ in the x, y, and z direction respectively. Temporally, we introduce a time step $\Delta t$ which allows us to propagate Maxwell's Equations forward in time by $\Delta t$ for every iteration.

For simplicity, non-relativistic equations will be introduced in this section, but they can easily be generalized to the relativistic versions which are implemented in modern PIC codes. Additionally, some of the equations will assume a 2D grid, but a 3D grid is similarly straightforward to generalize.

\subsection{Densities and Shape Factors}

When a simulation is initialized, all the particles will have a defined position and velocity. The charge density $\rho_{i,j}$ (for the cell at the $i^\text{th}$ and $j^\text{th}$ grid point in the x and y directions) is easy to compute -- it is simply the sum of all the charges $q_\alpha$ closest to grid point $(i,j)$ divided by the cell area: $\rho_{i,j} \equiv \frac{\sum_\alpha q_\alpha}{\Delta x \Delta y}$ (in 2D symmetry we additionally divide by 1 meter in the z direction to get the units right). The current density $\vec{J}_{i,j}$ can be obtained similarly -- $\vec{J}_{i,j} \equiv \frac{\sum_\alpha q_\alpha \vec{v}_\alpha}{\Delta x \Delta y}$. Assigning the densities to the nearest grid point in this manner is sensibly called Nearest Grid Point (NGP) by Birdsall and Langdon\cite{Birdsall_2004_PIC}.

Since the PIC approach contains many real particles in each macro particle, it is desired to smooth the macro particle densities throughout the cell(s). We can modify the individual density contributions of particles by a shape factor $S(\vec{r_\alpha} - \vec{r})$ that depends on a particle's location $\vec{r_\alpha}$ in relation to a grid point located at $\vec{r}$. This shape factor is normalized so that integrating it over the area of the simulation yields 1 to ensure the particle number is properly being conserved. The simplest improvement over NGP would be the \emph{top hat} shape factor (also called Cloud in Cell\cite{Birdsall_2004_PIC}) which assigns density contributions proportional to proximity of the nearest cells within ($\Delta x$,$\Delta y$). This has the shape of a uniform distribution and thus looks like a ``top hat'' in 1D. It is a $0^\text{th}$ order shape factor and can be represented by the following equation

\begin{equation}
	S_0(x) \equiv \begin{cases}
		1 & \text{if } \lvert x \rvert \leq 0.5 \Delta x \\
		0 & \text{otherwise}
	\end{cases} \label{eq:tophat}
\end{equation}
A further improvement will weigh the particles closer to a particular grid point higher than a particle further away. If this weighting is linear over an area ($2 \Delta x$, $2 \Delta y$), it is called the \emph{triangle} shape factor and reprsented by the following equation in 1D

\begin{equation}
	S_1(x) \equiv \begin{cases}
		1-\frac{\lvert x \rvert}{\Delta x} & \text{if } \lvert x \rvert \leq \Delta x \\
		0 & \text{otherwise}
	\end{cases} \label{eq:triangle}
\end{equation}
It turns out that the higher order shape factors $S_n(x)$ can be represented by convolutions of $S_0(x)$ 

\begin{figure}
	\centering 
	\includegraphics[width=0.6\linewidth]{planning/images/shape_functions.png}
	\caption{The top hat ($S_0(x)$), triangle ($S_1(x)$) and $3^\text{rd}$ order spline ($S_3(x)$) are plotted in 1D.}
	\label{fig:shape_factors}
\end{figure}

\begin{equation}
	S_n(x) \equiv \int_{-\infty}^\infty S_{n-1}(x')S_0(x-x') dx'
\end{equation}
the shape factors for $n \geq 2$ are commonly called n-splines. The third order spline is used in this work and weights particles over an area ($2 \Delta x$, $2 \Delta y$) and is represented in 1D by 

\begin{equation}
	b_3(x) = \begin{cases}
		\frac{1}{6}(8 - 12 \lvert \tilde{x} \rvert + 6 \tilde{x}^2 - \tilde{x}^3) & \text{if } 1 \leq \lvert \tilde{x} \rvert \leq 2 \\
		\frac{1}{6}(4 - 6 \tilde{x}^2 + 3 \tilde{x}^3) & \text{if } \lvert \tilde{x} \rvert \leq 1 \\
		0 & \text{otherwise}
	\end{cases}
\end{equation}
where $\tilde{x} \equiv x / \Delta x$ normalizes the position $x$. See \cref{fig:shape_factors} for a comparison of the three shape factors. 

\subsection{Field Solver and Particle Push}

\begin{figure}
	\centering 
	\includegraphics[width=0.8\linewidth]{planning/images/yee_grid.PNG}
	\caption{The ``Yee'' grid is depicted (left) where the electric and magnetic field components are staggered by half a cell. The fields, currents, position, and velocity make use of the staggered grid by leapfrog time integration (right). This picture was taken from the WarpX documentation}
	\label{fig:yee_grid}
\end{figure}

The PIC method is able to make efficient use of the second order accurate central difference approximation to compute derivatives. A simpler method like Euler integration is only first order accurate and will suffer in terms of accuracy. Higher order methods like 4th order Runge-Kutta have much higher computational costs in terms of operations per time step and memory consumption. The central difference scheme is accomplished by alternately calculating electric and magnetic fields, staggered by half a time step, in an approach called \emph{leapfrog integration}\cite{Birdsall_2004_PIC}. This can be seen in the right half of \cref{fig:yee_grid} where the calculations of $E$ and $J,B$ alternate in a ``leapfrog'' fashion. It turns out that this staggering also comes with some nice properties like automatically satisfying \cref{eq:gauss_magnetism}. By rearranging \cref{eq:faraday}, we can update the electric and magnetic fields through the following equations\cite{Arber_2015_PPCF}

\begin{align}
	\vec{E}^{n+\frac{1}{2}} &= \vec{E}^{n-\frac{1}{2}} + \Delta t (c^2 \nabla \times \vec{B}^n - \frac{1}{\epsilon_0} \vec{J}^n) \label{eq:E_update} \\
	\vec{B}^{n+1} &= \vec{B}^{n} - \Delta t (\nabla \times \vec{E}^{n+\frac{1}{2}}) \label{eq:B_update}
\end{align}
Given the electric and magnetic fields, we can now update the particle positions through \cref{eq:lorentz_pic}. The standard approach to update velocities and positions in a time-centered way would involve kinematics equations $v^{n+1/2} = v^{n-1/2} + a^n \Delta t$ and $x^{n+1} = x^{n} + v^{n_1/2} \Delta t$. The acceleration is both a function of $x$ and $v$ (\cref{eq:lorentz_pic}) and so cannot be calculating accurately in a time-centered way. This can be fixed through the application of the Boris rotation algorithm which removes the 

\begin{equation}
	\frac{v_{n + 1/2} - v_{n - 1/2}}{\Delta t} = \frac{q}{m}[E_n + v \times B]
\end{equation}

\subsection{EPOCH Code}

\begin{quote}
	Clearly Top-hat shape functions should never be used for laser-solid simulations.
\end{quote}

\subsection{Convergence Testing the EPOCH Code}


\section{Machine Learning}

\subsection{Simple Models}

\subsubsection{Polynomial Regression}

\subsubsection{Support Vector Regression}


\subsection{Advanced Models}

\subsubsection{Gaussian Process}

\subsection{Neural Networks}
\chapter{Let It End}

\lipsum[1]
\section{Please?}

\lipsum



\backmatter
% We use BIBTeX for the bibliography---you don't have to
\bibliographystyle{unsrt} % use your favorite BIBTeX style
% \nocite{*} % To display all refs, even uncited refs (useful when editting)
\bibliography{templatebib}

% If for some reason you are anti-BIBTeX, then you would use the
% following instead of the above:
%\begin{thebibliography}{99}
% ...
%\end{thebibliography}


% Note: GS 2010 requires bibliography/references _before_ the appendix
% if you believe their guidelines; however, conversations with GS
% staff suggests _they don't care_. Go figure. So do what you like.

\appendix
\chapter{Energy Conservation in EPOCH Particle-in-Cell Simulations Due to Finite Numbers of Particles}
This appendix focuses on unpublished work that was done jointly with Ricky Oropeza and Joseph Smith. My contributions to this project were primarily done as a pre-candidacy student.

\gls{PIC} simulations provide a useful but imperfect model of various plasma phenomena. In this work, the impact of the finite number of particles in a PIC simulation on the energy conservation is considered and explored through ultra-intense laser interactions with a thin, near solid density target in the \gls{TNSA} regime. 

\section{Background}

Explicit PIC codes tend to gain energy over time through what can be attributed as numerical errors. In this section, we consider which plasma and simulation parameters affect this numerical energy gain and derive various scalings. 

\subsection{Electric Field Fluctuations}

In PIC simulations, we compute the velocities at the next timestep through \cref{eq:v_update} which is dependent on the time-step $\Delta t$. Due to using a finite grid, approximating real particles with macro particles, and using a finite $\Delta t$, we will develop some errors in calculating the electric field $\delta E$. The corresponding force miscalculation $\delta F = q \delta E \Delta t$ would deliver an impulse $m \delta v$ and result in a velocity difference \cite{Hockney_1988_PIC} of

\begin{equation}
	\delta v = \frac{q}{m} \Delta t \delta E
\end{equation}
We can make an assumption that these field calculation errors will be randomly distributed which can be treated as a random walk in velocity space. If we consider $\Delta v$ as the total deviation of the calculated velocity from the true value, we should expect $\langle \Delta v \rangle = 0$ due to the symmetry of this random walk. However, the squared deviations on average will increase over time; for $n$ time-steps (each with the same random error $\delta E$), we would have 

\begin{equation}
	\langle \Delta v^2 \rangle = n \delta v ^2 = n \frac{q^2}{m^2} \Delta t^2 \delta E^2
\end{equation}
We can see that the average change in kinetic energy $\Delta KE \equiv \frac{1}{2} m \langle v^2 \rangle$ increases linearly with the number of time-steps $n$ \cite{Hockney_1988_PIC}. Additionally, since $\Delta KE \propto \frac{1}{m}$, the heavier particles (i.e. ions) can usually be neglected when examining the artificial heating \cite{Hockney_1988_PIC}. Hockney postulates a related expression \cite{Hockney_1971_JoCP} for $\Delta KE$ in another work as 

\begin{equation}
	\Delta KE \sim \frac{q^2}{m} \langle E^2 \rangle \tau_\text{corr} \Delta t
\end{equation}
where $\tau_\text{corr}$ can be identified with the period of plasma oscillations $\sim \omega_{p,e}^{-1}$. Then, expressing the charge of one electron macro-particle as $q = e \frac{N}{N_{mac}} = \frac{e n}{n_{mac}} = \frac{e n \Delta x^2}{n_{ppc}}$, where $\Delta x$ is the cell size and $n_\text{ppc}$ is the number of electron macro-particles per cell, the kinetic energy increase becomes

\begin{equation}
	\Delta KE = (\frac{e}{m_e}) \frac{e n \Delta x^2}{n_{ppc}} \langle E^2 \rangle \frac{2 \pi}{\omega_{pe}} \Delta t 
\end{equation}
Hockney uses a result from Chapter 8.2 of Montgomery and Tidman \cite{Montgomery_1964_Plasma} for the squared electric field fluctuations

\begin{equation} 
	\frac{\langle E^2 \rangle}{8 \pi} = \frac{k_B T}{2} \int \int_{-\infty}^{+\infty} \frac{d k_x d k_y}{(2 \pi)^2} \frac{1}{(1 + (k_x^2 + k_y^2) \lambda_D^2)} 
\end{equation}
which can be solved by letting $u = k \lambda_D$ where $k = \sqrt{k_x^2 + k_y^2}$ and integrating with respect to the polar area element $k \; dk \; d \phi$

\begin{equation}
	\langle E^2 \rangle = \frac{k_B T}{4 \pi \epsilon_0 \lambda_D^2} \text{Log}(1 + u_{max^2}) \label{eq:e2}
\end{equation}
Here, $u_{max} = k_\text{max} \lambda_D$ corresponds to the maximum wavenumber $k_{max} = \frac{2 \pi}{\Delta x}$ considered which is limited by the resolution. 

\subsection{Empirical Heating Estimates}

Using \cref{eq:debye,eq:omegape}, $\Delta KE$ can now be expressed as 

\begin{equation}
	\Delta KE = \frac{n_e^{3/2} \Delta x^2}{n_{ppc}} \text{Log}(1 + u_{max}^2) \label{eq:logarber}
\end{equation}
A more empirical estimate for the heating can be obtained by asserting a general scaling of the heating time $\tau_H \simeq \frac{n_\text{ppc}^\alpha}{\omega_{p,e}} \left(\frac{\lambda_{D,0}}{\Delta x}\right)^d$, where $d$ and $\alpha$ are constants that can be fit empirically through simulations. If we assert that the linear energy increase $\frac{dT}{dt} = T_0 / \tau_H$, we develop a formula (again using \cref{eq:debye,eq:omegape}) for the linear energy increase 

\begin{equation}
	\frac{dT_{eV}}{dt_{ps}} = C_H \frac{T_{0,eV}^{1 - d/2} \Delta x_{nm}^d n_{23}^{(d + 1)/2}}{n_\text{ppc}^\alpha} \label{eq:generalizedarber}
\end{equation}
and when $\alpha = 1$ and $d = 2$, we obtain eq. (30) from Arber et al. \cite{Arber_2015_PPCF}

\begin{equation}
	\frac{dT_{eV}}{dt_{ps}} = C_H \frac{\Delta x_{nm}^2 n_{23}^{3/2}}{n_\text{ppc}} \label{eq:arber}
\end{equation}
where $C_H$ is a constant determined by the shape function and the use of current smoothing. The cell size, number density, time, and temperature are expressed in nm, $10^{23} \text{cm}^{-3}$, ps, and eV due to being convenient units for PIC simulations. A more sophisticated empirical model could also account for two dimensionless timescales: $\omega_{p, e} \Delta t$ and $v_\text{th} \Delta t$, but Hockney \cite{Hockney_1971_JoCP} notes that these can be ignored by constraining $\omega_{p,e} \Delta t$ to be $\omega_{p,e} \Delta t = \text{min}((2 \lambda_D / \Delta x)^{-1}, 1)$ \cite{Hockney_1971_JoCP}.

\section{Methods}

\section{Conclusion}

\chapter{I don't feel so good}

\lipsum


\end{document}
